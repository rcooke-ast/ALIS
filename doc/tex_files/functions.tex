\section{Built-in functions}
\label{sec:functions}

There are several built-in functions that come as standard with
\alis. If this suite of functions is not sufficient, then you can (easily)
write your own and read it into \alis\ (for more details on how to
do this, refer to Section~\ref{sec:newfunctions}). Please, if you write
a function that you believe will be useful to the wider community,
let me know and I'll include it in the next release! It is worth noting
that all functions in the source code directory have the prefix
\texttt{alfunc\_<function name>.py}.

It is worth reiterating at this point that you do not have to specify
every parameter of the model. However, if you decide to not
set a parameter, \alis\ will use the default value, and will fix this
value.

In the following subsections, each built-in function is described
in alphabetical order in more detail. Below is an incomplete list of
the current model functions. To see the full list, please check the
source code --- all functions have a filename of the form
\texttt{alfunc\_<function\_name>.py}, where \texttt{<function\_name>.py}
includes one of the following:
\begin{multicols}{2}
\begin{itemize}
\item Afwhm
\item base
\item brokenpowerlaw
\item chebyshev
\item constant
\item gaussian
\item legendre
\item linear
\item polynomial
\item powerlaw
\item random
\item tophat
\item vfwhm
\item voigt
\item vsigma
\end{itemize}
\end{multicols}


\subsection{Afwhm}

The \texttt{Afwhm} function is a model that describes the
instrumental broadening as a Normal Distribution function
with standard deviation (in Angstroms) given by:
\begin{equation}
\sigma({\rm \AA})=2\sqrt{2\,\ln\,2}\,p_0
\end{equation}
where the description of this single parameter is:
\begin{enumerate}
\setcounter{enumi}{-1}
\item \textbf{value} --- The full-width at half-maximum in Angstroms (Default = 0.0).
\end{enumerate}
The corresponding keywords are:
\begin{itemize}
\item \textbf{blind} --- If you would like to blind yourself from this model, set \texttt{blind=True}.
\end{itemize}
This function is specifically designed for convolving the
model spectrum with the instrumental broadening function.
The convolution is performed with a fast fourier transform.


\subsection{base}

\textbf{\underline{You should not change anything in this file}}.
Moreover, this function should not be used for any model
definition. It contains the guts for defining a model, and as the
name suggests, it is the base from which all other models are
defined. The benefit of this file is that you can easily write your
own functions by importing some definitions that have already
been written for you!

\subsection{brokenpowerlaw}

The \texttt{brokenpowerlaw} model is given by the following equation:
\begin{equation}
{\rm model}=\frac{p_0}{(\lambda/p_{3})^{p_{1}}(1.0+(\lambda/p_{3})^{p_{4}})^{(p_{2}-p_{1})/p_{4}}}
\end{equation}
I'm not sure if (or where) this style of function has been defined before,
but the idea is that this model will have a power-law form at both blue
and red wavelengths, where the power-law index is different for the
blue and red. There are five parameters which are given
the names:

\begin{enumerate}
\setcounter{enumi}{-1}
\item \textbf{coefficient} --- A scaling coefficient for the model (Default = 0.0).
\item \textbf{blueindex} --- The power-law index for blue wavelengths (Default = 0.0).
\item \textbf{redindex} --- The power-law index for red wavelengths (Default = 0.0).
\item \textbf{location} --- The location of the break (Default = 5000.0).
\item \textbf{strength} --- The sharpness of the break. Higher values provide a sharper break, whereas lower values provide a more gradual transition (Default = 1.0).
\end{enumerate}

The corresponding keywords are:

\begin{itemize}
\item \textbf{specid} --- The set of specid's that this model should be applied to.
\item \textbf{blind} --- If you would like to blind yourself from this model, set \texttt{blind=True}.
\end{itemize}


\subsection{chebyshev}

The \texttt{chebyshev} model is a Chebyshev polynomial of the first kind:
\begin{equation}
{\rm model}=p_0 + p_1\lambda + p_2(2\lambda^2 - 1) + \ldots
\end{equation}
You can specify as many parameters as you like (well, up to 10 --- you can
only fit a Chebyshev polynomial of order 9 and below). The `downside' of
having an arbitrary number of parameters is that you cannot specify the
\texttt{parid}, \texttt{fixpar}, \texttt{limited}, or \texttt{limits} parameters. Although
these commands will work, the limits you place will be applied to all coefficients,
which may not be so bad, if you use the \texttt{scale} keyword to scale your
coefficients to be of the same magnitude. If you want the ability to limit your
polynomial coefficients, another way around this would be to write your own
Chebyshev function for the  order that interests you. Every coefficient is given
a non-unique \texttt{parid}:

\begin{enumerate}
\setcounter{enumi}{-1}
\item \textbf{coefficient} --- The coefficient for the $n^{\rm th}$ Chebyshev polynomial (Default = 0.0).
\end{enumerate}

The corresponding keywords are:

\begin{itemize}
\item \textbf{specid} --- The set of specid's that this model should be applied to.
\item \textbf{blind} --- If you would like to blind yourself from this model, set \texttt{blind=True}.
\item \textbf{scale} --- An array that scales each of the coefficients If you use this keyword, you
must provide an array  of comma-separated numbers, with the same number of elements as
coefficients, of the form: \texttt{scale=[1.0,0.1,0.001,1.0E-6]}. This example would yield a third
order Chebyshev polynomial of the first kind.
\end{itemize}


\subsection{constant}

The \texttt{constant} model is just that --- a constant. The model equation is given by:
\begin{equation}
{\rm model}=p_0
\end{equation}
where the description of this parameter is:
\begin{enumerate}
\setcounter{enumi}{-1}
\item \textbf{value} --- The value of the constant (Default = 1.0).
\end{enumerate}
The corresponding keywords are:
\begin{itemize}
\item \textbf{specid} --- The set of specid's that this model should be applied to.
\item \textbf{blind} --- If you would like to blind yourself from this model, set \texttt{blind=True}.
\end{itemize}


\subsection{gaussian}

The \texttt{gaussian} model is a Gaussian function of the form:
\begin{equation}
{\rm model}=p_0 \exp\left(-\frac{(\lambda-{\rm wave}\times(1.0+p_1))^2}{2.0p_{2}^{2}}\right)
\end{equation}
where the parameters are given by:
\begin{enumerate}
\setcounter{enumi}{-1}
\item \textbf{amplitude} --- The amplitude of the Gaussian feature (Default = 0.0).
\item \textbf{redshift} --- The emission redshift (Default = 0.0).
\item \textbf{dispersion} --- One standard deviation (in km s$^{-1}$) (Default = 100.0).
\end{enumerate}
The corresponding keywords are:
\begin{itemize}
\item \textbf{specid} --- The set of specid's that this model should be applied to.
\item \textbf{blind} --- If you would like to blind yourself from this model, set \texttt{blind=True}.
\item \textbf{wave} --- Specify the ion+wavelength (or just a numerical wavelength) of the transition. This is a required parameter since it is used in the model calculation.
\end{itemize}


\subsection{legendre}

The \texttt{legendre} model is a Legendre polynomial of the first kind:
\begin{equation}
{\rm model}=p_0 + p_1\lambda + p_2\frac{3\lambda^2 - 1}{2} + \ldots
\end{equation}
You can specify as many parameters as you like (well, up to 10 --- you can
only fit a Legendre polynomial of order 10 and below). The `downside' of
having an arbitrary number of parameters is that you cannot specify the
\texttt{parid}, \texttt{fixpar}, \texttt{limited}, or \texttt{limits} parameters. Although
these commands will work, the limits you place will be applied to all coefficients,
which may not be so bad, if you use the \texttt{scale} keyword to scale your
coefficients to be of the same magnitude. If you want the ability to limit your
polynomial coefficients, another way around this would be to write your own
Legendre function for the order that interests you. Every coefficient is given
a non-unique \texttt{parid}:

\begin{enumerate}
\setcounter{enumi}{-1}
\item \textbf{coefficient} --- The coefficient for the $n^{\rm th}$ Legendre polynomial (Default = 0.0).
\end{enumerate}

The corresponding keywords are:

\begin{itemize}
\item \textbf{specid} --- The set of specid's that this model should be applied to.
\item \textbf{blind} --- If you would like to blind yourself from this model, set \texttt{blind=True}.
\item \textbf{scale} --- An array that scales each of the coefficients If you use this keyword, you
must provide an array  of comma-separated numbers, with the same number of elements as
coefficients, of the form: \texttt{scale=[1.0,0.1,1.0E-3]}. This example would yield a second
order Legendre polynomial of the first kind.
\end{itemize}


\subsection{linear}

The \texttt{linear} function is a straight line. The model equation is given by:
\begin{equation}
{\rm model}=p_0 + p_1 \lambda
\end{equation}
where the description of these parameters are:
\begin{enumerate}
\setcounter{enumi}{-1}
\item \textbf{intercept} --- The value of the function when $\lambda=0$ (Default = 1.0).
\item \textbf{gradient} --- The slope of the line (Default = 0.0).
\end{enumerate}
The corresponding keywords are:
\begin{itemize}
\item \textbf{specid} --- The set of specid's that this model should be applied to.
\item \textbf{blind} --- If you would like to blind yourself from this model, set \texttt{blind=True}.
\end{itemize}


\subsection{polynomial}

The \texttt{polynomial} model is a standard polynomial of the form:
\begin{equation}
{\rm model}=p_0 + p_1\lambda + p_2\lambda^2 + \ldots
\end{equation}
You can specify as many parameters as you like. The `downside' of
having an arbitrary number of parameters is that you cannot specify the
\texttt{parid}, \texttt{fixpar}, \texttt{limited}, or \texttt{limits} parameters. Although
these commands will work, the limits you place will be applied to all coefficients,
which may not be so bad, if you use the \texttt{scale} keyword to scale your
coefficients to be of the same magnitude. If you want the ability to limit your
polynomial coefficients, another way around this would be to write your own
Polynomial function for the order that interests you. Every coefficient is given
a non-unique \texttt{parid}:

\begin{enumerate}
\setcounter{enumi}{-1}
\item \textbf{coefficient} --- The coefficient for the $n^{\rm th}$ term of the polynomial (Default = 0.0).
\end{enumerate}

The corresponding keywords are:

\begin{itemize}
\item \textbf{specid} --- The set of specid's that this model should be applied to.
\item \textbf{blind} --- If you would like to blind yourself from this model, set \texttt{blind=True}.
\item \textbf{scale} --- An array that scales each of the coefficients If you use this keyword, you
must provide an array  of comma-separated numbers, with the same number of elements as
coefficients, of the form: \texttt{scale=[10.0,0.1,1.0E-3]}. This example would yield a second
order polynomial (i.e. a quadratic).
\end{itemize}


\subsection{powerlaw}

The \texttt{powerlaw} function is a single powerlaw. The model equation is given by:
\begin{equation}
{\rm model}=p_0\lambda^{p_1}
\end{equation}
where the description of these parameters are:
\begin{enumerate}
\setcounter{enumi}{-1}
\item \textbf{coefficient} --- A multiplicative (or scaling) constant (Default = 0.0).
\item \textbf{index} --- The index of the powerlaw (Default = 0.0).
\end{enumerate}
The corresponding keywords are:
\begin{itemize}
\item \textbf{specid} --- The set of specid's that this model should be applied to.
\item \textbf{blind} --- If you would like to blind yourself from this model, set \texttt{blind=True}.
\end{itemize}


\subsection{random}

The \texttt{random} function forces the starting parameter value to be a random number drawn from a distribution specified by the `\texttt{command}' keyword argument. You will need to do some linking as
well, to tell the \texttt{random} function which variable to apply the randomly generated value to. For
example, if you know that the logarithm of the column density in your voigt profile is somewhere
between 13.0 and 14.0, you might specify the following model:

\vspace{0.3cm}
\begin{mdframed}[style=MyFrame]
model read\\
emission\\
constant 1.0CONST\\
absorption\\
voigt ion=1Ly\_a    13.2lra   redshift=0.47567    4.0   1.0E4TA\\
random 0.0lrb  command=uniform(13.0,14.0)\\
model end\\

\vspace{0.1cm}

\noindent
link read\\
lra(lrb) = lrb\\
link end\\
\end{mdframed}
\vspace{0.2cm}

The \texttt{uniform} distribution is just one of the allowed values. You could
also choose, for example, a normal distribution with centroid 13.5 and width
0.3. In this case, you would have \texttt{command=normal(13.5,0.3)}. A list of
all the permitted distributions are specified on the following website:

\url{http://docs.scipy.org/doc/numpy/reference/routines.random.html}

\textbf{\underline{NOTE:}} Be warned, placing a prior on a starting parameter's value should
be considered carefully; you should almost always use a uniform prior with two extreme
boundaries (that are considered as the limits of all possible values), unless you know what
you're doing.

\subsection{tophat}

The \texttt{tophat} function is a rectangular function. The model equation is given by:
\begin{equation}
  {\rm model}=\begin{cases}
    0, & \text{if $\lambda < p_1 - p_2/2$}.\\
    p_0, & \text{if $p_1 - p_2/2 \le \lambda < p_1 + p_2/2$}.\\
    0, & \text{if $\lambda \ge p_1 + p_2/2$}.
  \end{cases}
\end{equation}
where the description of these parameters are:
\begin{enumerate}
\setcounter{enumi}{-1}
\item \textbf{height} --- The amplitude of the tophat function in the specified interval (Default = 1.0).
\item \textbf{centroid} --- The centroid of the tophat function (Default = 0.0).
\item \textbf{width} --- The width of the tophat function (Default = 1.0).
\end{enumerate}
The corresponding keywords are:
\begin{itemize}
\item \textbf{specid} --- The set of specid's that this model should be applied to.
\item \textbf{blind} --- If you would like to blind yourself from this model, set \texttt{blind=True}.
\item \textbf{hstep} --- The step size to be used when calculating the partial derivative of the model with respect to the  \texttt{centroid} (should be of order \texttt{width}).
\item \textbf{wstep} --- The step size to be used when calculating the partial derivative of the model with respect to the  \texttt{width} (should be of order \texttt{width}).
\end{itemize}


\subsection{variable}

The \texttt{variable} function creates a dummy variable that can be used to connect two parameters through the \texttt{link} command. Suppose you know that a parameter is exactly a constant times another parameter (but you don't know what either parameter is, nor the cosntant scaling). For example, suppose you want to fit a two component Voigt model, where silicon and sulphur are tied to have the same relative abundance, but you don't know the column density of Si\,\textsc{ii} or S\,\textsc{ii} or the relative abundance. In this case, you could specify the model

\vspace{0.3cm}
\begin{mdframed}[style=MyFrame]
model read\\
emission\\
constant 1.0CONST\\
absorption\\
voigt ion=28Si\_II    13.2colsia   redshift=0.475686ra    4.0da   1.0E4TA\\
voigt ion=28Si\_II    13.0colsib   redshift=0.475705rb    3.0db   1.0E4TB\\
voigt ion=32S\_II    12.2colsa   redshift=0.475686ra    4.0da   1.0E4TA\\
voigt ion=32S\_II    12.0colsb   redshift=0.475705rb    3.0db   1.0E4TB\\
variable -1.0lnksis \\
model end\\

\vspace{0.1cm}

\noindent
link read\\
colsa(colsia,lnksis) = colsia+lnksis\\
colsb(colsib,lnksis) = colsib+lnksis\\
link end\\
\end{mdframed}
\vspace{0.2cm}

which forces the S\,\textsc{ii} column densities to be set by the Si\,\textsc{ii} column densities $+$ some fitted constant value called \texttt{lnksis}. Another example, is that you can use a variable to find the error on the summed column density for a given ion. Here is an example of that for Si:

\vspace{0.3cm}
\begin{mdframed}[style=MyFrame]
model read\\
emission\\
constant 1.0CONST\\
absorption\\
voigt ion=28Si\_II    13.2colsia   redshift=0.475686ra    4.0da   1.0E4TA\\
voigt ion=28Si\_II    13.0colsib   redshift=0.475705rb    3.0db   1.0E4TB\\
variable 13.5lnksis \\
model end\\

\vspace{0.1cm}

\noindent
link read\\
colsib(lnksis,colsia) = numpy.log10(10.0**lnksis - 10.0**colsia)\\
link end\\
\end{mdframed}
\vspace{0.2cm}

In this last case, \texttt{lnksis} is the total (summed) column density. \textbf{NOTE:} It is generally a good idea to specify \texttt{colsib} as the lowest column density component of the absorption.

\subsection{vfwhm}

The \texttt{vfwhm} function is a model that describes the
instrumental broadening as a Normal Distribution function
with standard deviation (in velocity) given by:
\begin{equation}
\sigma(v)=2\sqrt{2\,\ln\,2}\,p_0
\end{equation}
where the description of this single parameter is:
\begin{enumerate}
\setcounter{enumi}{-1}
\item \textbf{value} --- The velocity full-width at half-maximum (Default = 0.0).
\end{enumerate}
The corresponding keywords are:
\begin{itemize}
\item \textbf{blind} --- If you would like to blind yourself from this model, set \texttt{blind=True}.
\end{itemize}
This function is specifically designed for convolving the
model spectrum with the instrumental broadening function.
The convolution is performed with a fast fourier transform.


\subsection{voigt}

The \texttt{voigt} function describes the absorption line profile for a group
of atoms that obey a Maxwell-Boltzmann distribution. The functional form
is the convolution of the intrinsic line profile (i.e. a Lorentzian) with a
Maxwell-Boltzmann distribution. This profile involves a very numerically
expensive calculation. To avoid this time-consuming operation, \alis\
uses a tabulated form of the Voigt profile which expands the following
functional form into a Taylor series. The implementation that is used in
\alis\ is the same as that used in \texttt{vpfit}, prepared by Julian King
(which is an extension from the work by others), and is correct to
within 1 part in $10^5$. For reference, the functional form of the Voigt
profile is as follows:
\begin{equation}
{\rm model}=I(\lambda)_0 e^{-p_0\,\sigma_{\lambda}}
\end{equation}
where $I(\lambda)_0$ is the continuum intensity, and $\sigma_{\lambda}$ is given by:
\begin{equation}
\sigma_{\lambda}=a_{0}\,H(a,x)
\end{equation}
For a given transition, $a_{0}$ contains the atomic parameters
and $H(a,x)$ is known as the Voigt integral. These are given by:
\begin{eqnarray}
H(a,x) & = & \frac{a}{\pi}\int_{-\infty}^{\infty}\frac{\exp(-y^2)\,{\rm d}y}{(x-y)^2+a^2}
\\           
\nonumber
\\
a_{0} & = &\frac{\sqrt{\pi}\,e^2}{m_{e}\,c^{2}}\frac{f}{\Delta\nu_{D}}
\end{eqnarray}
where the oscillator strength, $f$, is read in from the \texttt{atomic.xml} file.
The damping parameter of  the intrinsic line shape, $a$, and the
Doppler frequency, $\Delta\nu_{D}$, have the form
\begin{eqnarray}
a & = &\frac{\Gamma}{4\pi\,\Delta\nu_{D}}
\\
\nonumber
\\
\Delta\nu_{D} & = & \frac{b}{\lambda_{0}} \,\, = \,\, \frac{1}{\lambda_{0}}\sqrt{b_{th}^{2}+p_{2}^{2}},
\end{eqnarray}
where $\Gamma$ and $\lambda_{0}$ are respectively the
transition rate and the rest wavelength of the transition
(both are read in from \texttt{atomic.xml}). To convert the
rest wavelength into the observed frame, the observed
wavelength if $\lambda_{\rm obs}=\lambda_{0}(1+p_{1})$.
$b_{th}$ is the thermal Doppler parameter (describing the 
thermal broadening of the line profile) which is given by
\begin{eqnarray}
b_{th} & = & \sqrt{\frac{2\,k\,p_{3}}{m_{atom}}}
\end{eqnarray}
Finally, the dimensionless parameter $x$ in the Voigt integral
describes the frequency offset from the line centre, in units 
of the Doppler frequency,
\begin{eqnarray}
x & = & \frac{\nu-\nu_{0}}{\Delta\nu_{D}}.
\end{eqnarray}
and $y$ is the convolution parameter. If you are estimating a
possible variation in the fine-structure constant, the rest
wavenumber, $w_{0}\equiv1/\lambda_{0}$ is shifted by:
\begin{eqnarray}
w_{z} = w_{0} + q([1 + p_{4}]^{2} - 1)
\end{eqnarray}
where $q$ is as a number that describes how sensitive a given
atomic transition is to changes in the fine-structure constant, and
is read in from the \texttt{atomic.xml} file (for a small handful of
transitions).
\textbf{DESCRIBE VARIATION IN PROTON-TO-ELECTRON MASS RATIO HERE}
In summary, the parameters are given by:
\begin{enumerate}
\setcounter{enumi}{-1}
\item \textbf{ColDens} --- The column density (in cm$^{-2}$), where by default the logarithmic value of the column density is calculated, but see keywords (Default = 8.1).
\item \textbf{redshift} --- The absorption redshift (Default = 0.0).
\item \textbf{bturb} --- The turbulent Doppler parameter (in km s$^{-1}$) (Default = 7.0).
\item \textbf{temperature} --- The kinetic temperature of the gas (in K) (Default = 100.0).
\item \textbf{DELTAa/a} --- The relative variation of the fine-structure constant (Default = 0.0).
\item \textbf{DELTAmu/mu} --- The relative variation of the proton-to-electron mass ratio (Default = 0.0).
\end{enumerate}
The corresponding keywords are:
\begin{itemize}
\item \textbf{specid} --- The set of specid's that this model should be applied to.
\item \textbf{blind} --- If you would like to blind yourself from this model, set \texttt{blind=True}.
\item \textbf{ion} --- Specify the element+ionization stage (separated by an underscore). The
first letter of the element should be a uppercase letter. This is a required parameter since it is
used in the model calculation.
\item \textbf{logn} --- A True or False argument will respectively calculate the log or linear
value of the column density for this one model. For weak lines with large errors, you should
always use linear (i.e. \texttt{logn=False}), it will give you a more reliable estimate of your
errors. Any column density that is well-measured (with a low error), can use either, but in
this case, I recommend using the logarithmic value of the column density (i.e. \texttt{logn=True}).
\end{itemize}


\subsection{vsigma}

The \texttt{vsigma} function is a model that describes the
instrumental broadening as a Normal Distribution function
with standard deviation given by the only parameter of this
model, $p_0$, which is called \texttt{value}.
The keywords for this function are:
\begin{itemize}
\item \textbf{blind} --- If you would like to blind yourself from this model, set \texttt{blind=True}.
\end{itemize}
This function is specifically designed for convolving the
model spectrum with the instrumental broadening function.
The convolution is performed with a fast fourier transform.

