\section{How \alis\ works}
\label{sec:description}

This section is designed to give you a brief overview of how
the software works behind the scenes, and provides some
suggestions for how you can change/modify aspects of
the code (without harming \alis!).

In general, you will need to create a \dmod\ model file
(e.g. \texttt{myfirstfit.mod}) which tells \alis\ how to run.
You will then execute the code by typing the following
on a command line\footnote{Yes, a GUI is on the cards... eventually.}:

\vspace{0.3cm}
\begin{mdframed}[style=MyFrame]
run\_alis myfirstfit.mod
\end{mdframed}
\vspace{0.2cm}

From here, \alis\ will read in the default settings, change these
settings as you specify in the \texttt{.mod} file, load the data,
load the model, fit the model to the data, and provide a bunch
of output (which you can turn on/off as you require). I'll now
briefly describe each of these separately in the following sections.

At this point, I wish to acknowledge that I have used the \textsc{mpfit}
software package \citep{Mar09}, which I have modified for the purposes
of \alis. \textsc{mpfit} was originally called \textsc{lmfit} when it was a
\textsc{fortran} code long, long ago. From here, it was rewritten in
\textsc{idl} by \textbf{Craig Markwardt} (and received the new name
\textsc{mpfit}), and was then converted into \python\ by \textbf{Mark Rivers}.
Finally, \textbf{Sergey Koposov} updated the \python\ version of \textsc{mpfit}
to use \textsc{numpy}, rather than its predecessor (\textsc{numeric}).
The current (and all preceding versions) of \textsc{mpfit} uses a
Levenberg-Marquardt least-squares minimisation algorithm to
derive the model parameters that best fit the data (i.e. the parameters
that minimise the difference between the data and the model, weighted
by the error in the data). The underlying assumption here is that the
data are normally-distributed. The relevant updates that I have
made to \textsc{mpfit} includes CPU multiprocessing, a more
informative screen output that is suited to the \alis\ software, and
a few (almost) inconsequential fixes.

\subsection{The settings.alis file}\label{sec:settings}

This file contains the default settings that \alis\ needs in order to
run. I do not recommend that you change the parameters in this
file (since you can change each and every setting in your
\dmod\ file). But if you insist on changing something,
here is a brief description of what you can/can't do. This
file contains a three-argument command. The first argument
is just a name (or identifier) that \alis\ uses. The second
argument is the name given to a particular setting. The
third argument is the value of this setting.

The first argument can take the following values:
\texttt{run} (which tells \alis\ how to run), 
\texttt{chisq} (to set the details of the chi-squared minimisation, including convergence properties), 
\texttt{plot} (to describe how the data/model should be prepared for plotting), 
\texttt{out} (the information you want to output), 
\texttt{sim} (to perform Monte Carlo simulations), 
and
\texttt{generate} (to generate fake data).

Table~\ref{tab:settings} provides a summary of the
permitted arguments/values\footnote{Unfortunately, this list is probably incomplete, but has the most useful commands that you're likely to need -- feel free to dig into the code to find the other arguments as I update the code!}, and includes a description
for each of the parameters. As mentioned earlier, you
can also change these parameters manually for each
fit that you perform, by including the same three argument
commands (separated by whitespace or tabs) in your
\texttt{.mod} input model file.

\begin{center}
\begin{longtable}{p{1.5cm}p{2.5cm}p{2.0cm}p{2.0cm}p{5.0cm}}
\caption[]{\textsc{The allowed parameter settings for \alis}}\label{tab:settings} \\

\toprule
\multicolumn{1}{p{1.5cm}}{Arg. 1} &
\multicolumn{1}{p{2.5cm}}{Arg. 2} &
\multicolumn{1}{p{2.0cm}}{Default} &
\multicolumn{1}{p{2.0cm}}{Allowed} &
\multicolumn{1}{p{5.0cm}}{Description}\\
&
&
\multicolumn{1}{p{2.0cm}}{Value} &
\multicolumn{1}{p{2.0cm}}{Values} & \\ \midrule
\endfirsthead

\multicolumn{5}{c}%
{{\tablename\ \thetable{} -- continued from previous page}} \\
\toprule
\multicolumn{1}{p{1.5cm}}{Arg. 1} &
\multicolumn{1}{p{2.5cm}}{Arg. 2} &
\multicolumn{1}{p{2.0cm}}{Default} &
\multicolumn{1}{p{2.0cm}}{Allowed} &
\multicolumn{1}{p{5.0cm}}{Description}\\
&
&
\multicolumn{1}{p{2.0cm}}{Value} &
\multicolumn{1}{p{2.0cm}}{Values} &
\\ \midrule
\endhead

\bottomrule \multicolumn{5}{R}{Continued on next page...} \\
\endfoot

\bottomrule \multicolumn{5}{p{13.0cm}}{where int, float and str respectively refer to an integer, floating point, and character string value.}
\endlastfoot

\multicolumn{1}{p{1.5cm}}{\texttt{run}} &
\multicolumn{1}{p{2.5cm}}{\texttt{atomic}} &
\multicolumn{1}{p{2.0cm}}{atomic.xml} &
\multicolumn{1}{p{2.0cm}}{str} &
\multicolumn{1}{p{5.0cm}}{The name of the table that contains the relevant atomic transitions (must be a VOTable with suffix \texttt{.xml})} \\ \midrule

\multicolumn{1}{p{1.5cm}}{\texttt{run}} &
\multicolumn{1}{p{2.5cm}}{\texttt{bintype}} &
\multicolumn{1}{p{2.0cm}}{km/s} &
\multicolumn{1}{p{2.0cm}}{km/s, A} &
\multicolumn{1}{p{5.0cm}}{Set the type of pixels to use -- If pixels have constant velocity use ``km/s'', if pixels have constant Angstroms use ``A'' } \\ \midrule

\multicolumn{1}{p{1.5cm}}{\texttt{run}} &
\multicolumn{1}{p{2.5cm}}{\texttt{blind}} &
\multicolumn{1}{p{2.0cm}}{True} &
\multicolumn{1}{p{2.0cm}}{True, False} &
\multicolumn{1}{p{5.0cm}}{Run a blind analysis -- In order to print out the best-fitting model, this must be set to False} \\ \midrule

\multicolumn{1}{p{1.5cm}}{\texttt{run}} &
\multicolumn{1}{p{2.5cm}}{\texttt{capvalue}} &
\multicolumn{1}{p{2.0cm}}{None} &
\multicolumn{1}{p{2.0cm}}{None, float $>0.0$} &
\multicolumn{1}{p{5.0cm}}{If not None, set the model value to be the capvalue everywhere the model exceeds the capvalue} \\ \midrule

\multicolumn{1}{p{1.5cm}}{\texttt{run}} &
\multicolumn{1}{p{2.5cm}}{\texttt{convergence}} &
\multicolumn{1}{p{2.0cm}}{False} &
\multicolumn{1}{p{2.0cm}}{True, False} &
\multicolumn{1}{p{5.0cm}}{Run a convergence check} \\ \midrule

\multicolumn{1}{p{1.5cm}}{\texttt{run}} &
\multicolumn{1}{p{2.5cm}}{\texttt{convnostop}} &
\multicolumn{1}{p{2.0cm}}{False} &
\multicolumn{1}{p{2.0cm}}{True, False} &
\multicolumn{1}{p{5.0cm}}{Continue to reduce the chi-squared stopping criteria until the parameters have converged} \\ \midrule

\multicolumn{1}{p{1.5cm}}{\texttt{run}} &
\multicolumn{1}{p{2.5cm}}{\texttt{convcriteria}} &
\multicolumn{1}{p{2.0cm}}{0.2} &
\multicolumn{1}{p{2.0cm}}{float $>0.0$} &
\multicolumn{1}{p{5.0cm}}{Convergence criteria in units of the parameter error} \\ \midrule

\multicolumn{1}{p{1.5cm}}{\texttt{run}} &
\multicolumn{1}{p{2.5cm}}{\texttt{datatype}} &
\multicolumn{1}{p{2.0cm}}{default} &
\multicolumn{1}{p{2.0cm}}{string} &
\multicolumn{1}{p{5.0cm}}{Specify the type of data being read in (the default is the IRAF standard). \texttt{HIRESredux} will read in data reduced by \textsc{HIRES\_redux}. \texttt{UVESpopler} will read in data combined with \textsc{UVES\_popler}.} \\ \midrule

\multicolumn{1}{p{1.5cm}}{\texttt{run}} &
\multicolumn{1}{p{2.5cm}}{\texttt{limpar}} &
\multicolumn{1}{p{2.0cm}}{False} &
\multicolumn{1}{p{2.0cm}}{True, False} &
\multicolumn{1}{p{5.0cm}}{If an initial parameter's value is outside the model limits, set the initial value to the limit. Otherwise, an error is raised.} \\ \midrule

\multicolumn{1}{p{1.5cm}}{\texttt{run}} &
\multicolumn{1}{p{2.5cm}}{\texttt{ncpus}} &
\multicolumn{1}{p{2.0cm}}{-1} &
\multicolumn{1}{p{2.0cm}}{int} &
\multicolumn{1}{p{5.0cm}}{Number of CPUs to use (-1 means all bar one CPU, -2 means all bar two CPUs)} \\ \midrule

\multicolumn{1}{p{1.5cm}}{\texttt{run}} &
\multicolumn{1}{p{2.5cm}}{\texttt{ngpus}} &
\multicolumn{1}{p{2.0cm}}{0} &
\multicolumn{1}{p{2.0cm}}{int $\ge0$} &
\multicolumn{1}{p{5.0cm}}{Number of processes to send to the GPU/GPUs (must be $\ge0$)} \\ \midrule

\multicolumn{1}{p{1.5cm}}{\texttt{run}} &
\multicolumn{1}{p{2.5cm}}{\texttt{nsubpix}} &
\multicolumn{1}{p{2.0cm}}{5} &
\multicolumn{1}{p{2.0cm}}{int $\ge0$} &
\multicolumn{1}{p{5.0cm}}{Number of sub-pixels per 1 standard deviation to interpolate all models over -- higher values give higher precision} \\ \midrule

\multicolumn{1}{p{1.5cm}}{\texttt{run}} &
\multicolumn{1}{p{2.5cm}}{\texttt{nsubmin}} &
\multicolumn{1}{p{2.0cm}}{5} &
\multicolumn{1}{p{2.0cm}}{int $\ge0$} &
\multicolumn{1}{p{5.0cm}}{Minimum number of sub-pixels per 1 pixel to interpolate all models over -- higher values give higher precision} \\ \midrule

\multicolumn{1}{p{1.5cm}}{\texttt{run}} &
\multicolumn{1}{p{2.5cm}}{\texttt{nsubmax}} &
\multicolumn{1}{p{2.0cm}}{21} &
\multicolumn{1}{p{2.0cm}}{int $\ge0$} &
\multicolumn{1}{p{5.0cm}}{Maximum number of sub-pixels per 1 pixel to interpolate all models over -- higher values give higher precision} \\ \midrule

\multicolumn{1}{p{1.5cm}}{\texttt{run}} &
\multicolumn{1}{p{2.5cm}}{\texttt{renew\_subpix}} &
\multicolumn{1}{p{2.0cm}}{True} &
\multicolumn{1}{p{2.0cm}}{True, False} &
\multicolumn{1}{p{5.0cm}}{If True, the subpixellation will be calculated after every iteration} \\ \midrule

\multicolumn{1}{p{1.5cm}}{\texttt{run}} &
\multicolumn{1}{p{2.5cm}}{\texttt{warn\_subpix}} &
\multicolumn{1}{p{2.0cm}}{100} &
\multicolumn{1}{p{2.0cm}}{int $\ge0$} &
\multicolumn{1}{p{5.0cm}}{If the number of sub-pixels exceeds this amount, the user will be warned} \\ \midrule

\multicolumn{1}{p{1.5cm}}{\texttt{chisq}} &
\multicolumn{1}{p{2.5cm}}{\texttt{atol}} &
\multicolumn{1}{p{2.0cm}}{0.001} &
\multicolumn{1}{p{2.0cm}}{float $\ge0.0$} &
\multicolumn{1}{p{5.0cm}}{Termination criteria -- this measures the absolute change desired in the sum of squares} \\ \midrule

\multicolumn{1}{p{1.5cm}}{\texttt{chisq}} &
\multicolumn{1}{p{2.5cm}}{\texttt{ftol}} &
\multicolumn{1}{p{2.0cm}}{1.0E-10} &
\multicolumn{1}{p{2.0cm}}{float $\ge0.0$} &
\multicolumn{1}{p{5.0cm}}{Termination criteria -- this measures the relative error desired in the sum of squares} \\ \midrule

\multicolumn{1}{p{1.5cm}}{\texttt{chisq}} &
\multicolumn{1}{p{2.5cm}}{\texttt{gtol}} &
\multicolumn{1}{p{2.0cm}}{1.0E-10} &
\multicolumn{1}{p{2.0cm}}{float $\ge0.0$} &
\multicolumn{1}{p{5.0cm}}{Termination criteria -- this measures the orthogonality desired between the function vector and the columns of the Jacobian} \\ \midrule

\multicolumn{1}{p{1.5cm}}{\texttt{chisq}} &
\multicolumn{1}{p{2.5cm}}{\texttt{xtol}} &
\multicolumn{1}{p{2.0cm}}{1.0E-10} &
\multicolumn{1}{p{2.0cm}}{float $\ge0.0$} &
\multicolumn{1}{p{5.0cm}}{Termination criteria -- this measures the relative error desired in the approximate solution} \\ \midrule

\multicolumn{1}{p{1.5cm}}{\texttt{chisq}} &
\multicolumn{1}{p{2.5cm}}{\texttt{fstep}} &
\multicolumn{1}{p{2.0cm}}{20.0} &
\multicolumn{1}{p{2.0cm}}{float $>0.0$} &
\multicolumn{1}{p{5.0cm}}{Factor above machine-precision to use for step size} \\ \midrule

\multicolumn{1}{p{1.5cm}}{\texttt{chisq}} &
\multicolumn{1}{p{2.5cm}}{\texttt{maxiter}} &
\multicolumn{1}{p{2.0cm}}{20000} &
\multicolumn{1}{p{2.0cm}}{int $>1$} &
\multicolumn{1}{p{5.0cm}}{Maximum number of iterations before giving up} \\ \midrule

\multicolumn{1}{p{1.5cm}}{\texttt{chisq}} &
\multicolumn{1}{p{2.5cm}}{\texttt{miniter}} &
\multicolumn{1}{p{2.0cm}}{1} &
\multicolumn{1}{p{2.0cm}}{int $>1$} &
\multicolumn{1}{p{5.0cm}}{Minimum number of iterations before checking convergence criteria} \\ \midrule

\multicolumn{1}{p{1.5cm}}{\texttt{plot}} &
\multicolumn{1}{p{2.5cm}}{\texttt{fits}} &
\multicolumn{1}{p{2.0cm}}{True} &
\multicolumn{1}{p{2.0cm}}{True, False} &
\multicolumn{1}{p{5.0cm}}{Plot the best-fitting model with the data?} \\ \midrule

\multicolumn{1}{p{1.5cm}}{\texttt{plot}} &
\multicolumn{1}{p{2.5cm}}{\texttt{residuals}} &
\multicolumn{1}{p{2.0cm}}{False} &
\multicolumn{1}{p{2.0cm}}{True, False} &
\multicolumn{1}{p{5.0cm}}{Plot the residuals for the best-fitting model?} \\ \midrule

\multicolumn{1}{p{1.5cm}}{\texttt{plot}} &
\multicolumn{1}{p{2.5cm}}{\texttt{only}} &
\multicolumn{1}{p{2.0cm}}{False} &
\multicolumn{1}{p{2.0cm}}{True, False} &
\multicolumn{1}{p{5.0cm}}{Don't fit the data, just plot the input data and model} \\ \midrule

\multicolumn{1}{p{1.5cm}}{\texttt{plot}} &
\multicolumn{1}{p{2.5cm}}{\texttt{pages}} &
\multicolumn{1}{p{2.0cm}}{all} &
\multicolumn{1}{p{2.0cm}}{int, all} &
\multicolumn{1}{p{5.0cm}}{Which pages to plot? Either provide comma-separated integers or the text string `all'} \\ \midrule

\multicolumn{1}{p{1.5cm}}{\texttt{plot}} &
\multicolumn{1}{p{2.5cm}}{\texttt{dims}} &
\multicolumn{1}{p{2.0cm}}{3x3} &
\multicolumn{1}{p{2.0cm}}{(int)x(int)} &
\multicolumn{1}{p{5.0cm}}{Specify the plotting dimensions (ROWSxCOLUMNS)} \\ \midrule

\multicolumn{1}{p{1.5cm}}{\texttt{plot}} &
\multicolumn{1}{p{2.5cm}}{\texttt{fitregions}} &
\multicolumn{1}{p{2.0cm}}{False} &
\multicolumn{1}{p{2.0cm}}{True, False} &
\multicolumn{1}{p{5.0cm}}{Indicate the regions of data that are being used in the chi-squared minimation (i.e. that specified by fitrange)?} \\ \midrule

\multicolumn{1}{p{1.5cm}}{\texttt{plot}} &
\multicolumn{1}{p{2.5cm}}{\texttt{ticks}} &
\multicolumn{1}{p{2.0cm}}{True} &
\multicolumn{1}{p{2.0cm}}{True, False} &
\multicolumn{1}{p{5.0cm}}{Plot tick marks above the spectrum to indicate model components?} \\ \midrule

\multicolumn{1}{p{1.5cm}}{\texttt{plot}} &
\multicolumn{1}{p{2.5cm}}{\texttt{ticklabels}} &
\multicolumn{1}{p{2.0cm}}{False} &
\multicolumn{1}{p{2.0cm}}{True, False} &
\multicolumn{1}{p{5.0cm}}{Plot labels above the tick marks to identify model components?} \\ \midrule

\multicolumn{1}{p{1.5cm}}{\texttt{out}} &
\multicolumn{1}{p{2.5cm}}{\texttt{covar}} &
\multicolumn{1}{p{2.0cm}}{``''} &
\multicolumn{1}{p{2.0cm}}{str} &
\multicolumn{1}{p{5.0cm}}{Output the covariance matrix, to a file with name given by third argument (No output if the argument is ``'')} \\ \midrule

\multicolumn{1}{p{1.5cm}}{\texttt{out}} &
\multicolumn{1}{p{2.5cm}}{\texttt{convtest}} &
\multicolumn{1}{p{2.0cm}}{``''} &
\multicolumn{1}{p{2.0cm}}{str} &
\multicolumn{1}{p{5.0cm}}{Output the details of the convergence test, to a file with name given by third argument (No output if the argument is ``'')} \\ \midrule

\multicolumn{1}{p{1.5cm}}{\texttt{out}} &
\multicolumn{1}{p{2.5cm}}{\texttt{fits}} &
\multicolumn{1}{p{2.0cm}}{False} &
\multicolumn{1}{p{2.0cm}}{True, False} &
\multicolumn{1}{p{5.0cm}}{Output the best fitting model fits?} \\ \midrule

\multicolumn{1}{p{1.5cm}}{\texttt{out}} &
\multicolumn{1}{p{2.5cm}}{\texttt{onefits}} &
\multicolumn{1}{p{2.0cm}}{False} &
\multicolumn{1}{p{2.0cm}}{True, False} &
\multicolumn{1}{p{5.0cm}}{Output the best fitting models to a single fits file?} \\ \midrule

\multicolumn{1}{p{1.5cm}}{\texttt{out}} &
\multicolumn{1}{p{2.5cm}}{\texttt{model}} &
\multicolumn{1}{p{2.0cm}}{True} &
\multicolumn{1}{p{2.0cm}}{True, False} &
\multicolumn{1}{p{5.0cm}}{Output the best fitting model parameters?} \\ \midrule

\multicolumn{1}{p{1.5cm}}{\texttt{out}} &
\multicolumn{1}{p{2.5cm}}{\texttt{overwrite}} &
\multicolumn{1}{p{2.0cm}}{False} &
\multicolumn{1}{p{2.0cm}}{True, False} &
\multicolumn{1}{p{5.0cm}}{Overwrite existing files when writing out?} \\ \midrule

\multicolumn{1}{p{1.5cm}}{\texttt{out}} &
\multicolumn{1}{p{2.5cm}}{\texttt{plots}} &
\multicolumn{1}{p{2.0cm}}{``''} &
\multicolumn{1}{p{2.0cm}}{str} &
\multicolumn{1}{p{5.0cm}}{Save the output plots to a pdf file, to a file with name given by third argument (No output if the argument is ``'')} \\ \midrule

\multicolumn{1}{p{1.5cm}}{\texttt{out}} &
\multicolumn{1}{p{2.5cm}}{\texttt{sm}} &
\multicolumn{1}{p{2.0cm}}{False} &
\multicolumn{1}{p{2.0cm}}{True, False} &
\multicolumn{1}{p{5.0cm}}{Generate a SuperMongo plotting script?} \\ \midrule

\multicolumn{1}{p{1.5cm}}{\texttt{out}} &
\multicolumn{1}{p{2.5cm}}{\texttt{reletter}} &
\multicolumn{1}{p{2.0cm}}{False} &
\multicolumn{1}{p{2.0cm}}{True, False} &
\multicolumn{1}{p{5.0cm}}{Set to True if you want \alis\ to reletter the tied/fixed parameters starting from `A'} \\ \midrule

\multicolumn{1}{p{1.5cm}}{\texttt{out}} &
\multicolumn{1}{p{2.5cm}}{\texttt{verbose}} &
\multicolumn{1}{p{2.0cm}}{2} &
\multicolumn{1}{p{2.0cm}}{0, 1, 2} &
\multicolumn{1}{p{5.0cm}}{Level of screen output (0 is for no screen output, 1 is low level output, 2 is output everything)} \\ \midrule

\multicolumn{1}{p{1.5cm}}{\texttt{sim}} &
\multicolumn{1}{p{2.5cm}}{\texttt{random}} &
\multicolumn{1}{p{2.0cm}}{None} &
\multicolumn{1}{p{2.0cm}}{int $> 0$} &
\multicolumn{1}{p{5.0cm}}{Number of random simulations to perform} \\ \midrule

\multicolumn{1}{p{1.5cm}}{\texttt{sim}} &
\multicolumn{1}{p{2.5cm}}{\texttt{perturb}} &
\multicolumn{1}{p{2.0cm}}{None} &
\multicolumn{1}{p{2.0cm}}{int $> 0$} &
\multicolumn{1}{p{5.0cm}}{Number of simulations to perform where the starting parameters are perturbed} \\ \midrule

\multicolumn{1}{p{1.5cm}}{\texttt{sim}} &
\multicolumn{1}{p{2.5cm}}{\texttt{systematics}} &
\multicolumn{1}{p{2.0cm}}{False} &
\multicolumn{1}{p{2.0cm}}{True, False} &
\multicolumn{1}{p{5.0cm}}{In addition to running random simulations, do you want to run systematics simulations? \texttt{sim+random} must be $> 0$ to run systematics simulations.} \\ \midrule

\multicolumn{1}{p{1.5cm}}{\texttt{sim}} &
\multicolumn{1}{p{2.5cm}}{\texttt{beginfrom}} &
\multicolumn{1}{p{2.0cm}}{``''} &
\multicolumn{1}{p{2.0cm}}{str} &
\multicolumn{1}{p{5.0cm}}{An input file called \texttt{<filename>.mod.out} that contains the starting parameters and errors for the simulations (\textsc{note}: you will need to also output the corresponding covariance matrix for these parameters, with a filename \texttt{<filename>.mod.covar})} \\ \midrule

\multicolumn{1}{p{1.5cm}}{\texttt{sim}} &
\multicolumn{1}{p{2.5cm}}{\texttt{startid}} &
\multicolumn{1}{p{2.0cm}}{0} &
\multicolumn{1}{p{2.0cm}}{int $\ge 0$} &
\multicolumn{1}{p{5.0cm}}{A starting ID label for the Monte Carlo simulations. This will be incremented by 1 for each new simulation until \texttt{sim+random} simulations have been performed.} \\ \midrule

\multicolumn{1}{p{1.5cm}}{\texttt{sim}} &
\multicolumn{1}{p{2.5cm}}{\texttt{systmodule}} &
\multicolumn{1}{p{2.0cm}}{None} &
\multicolumn{1}{p{2.0cm}}{str, None} &
\multicolumn{1}{p{5.0cm}}{If the user writes their own module to deal with systematics, specify the name of this module here. None will use the default, built-in systematics} \\ \midrule

\multicolumn{1}{p{1.5cm}}{\texttt{sim}} &
\multicolumn{1}{p{2.5cm}}{\texttt{newstart}} &
\multicolumn{1}{p{2.0cm}}{True} &
\multicolumn{1}{p{2.0cm}}{True, False} &
\multicolumn{1}{p{5.0cm}}{Generate a new set of starting parameters from the best-fitting covariance matrix?} \\ \midrule

\multicolumn{1}{p{1.5cm}}{\texttt{sim}} &
\multicolumn{1}{p{2.5cm}}{\texttt{dirname}} &
\multicolumn{1}{p{2.0cm}}{sims} &
\multicolumn{1}{p{2.0cm}}{str} &
\multicolumn{1}{p{5.0cm}}{Name of the folder to dump the output from the Monte Carlo runs} \\ \midrule

\multicolumn{1}{p{1.5cm}}{\texttt{sim}} &
\multicolumn{1}{p{2.5cm}}{\texttt{edgecut}} &
\multicolumn{1}{p{2.0cm}}{4.0} &
\multicolumn{1}{p{2.0cm}}{float $\ge0.0$} &
\multicolumn{1}{p{5.0cm}}{Number of standard deviations (based on instrumental FWHM) to reject from generated data (due to edge effects)} \\ \midrule

\multicolumn{1}{p{1.5cm}}{\texttt{generate}} &
\multicolumn{1}{p{2.5cm}}{\texttt{data}} &
\multicolumn{1}{p{2.0cm}}{False} &
\multicolumn{1}{p{2.0cm}}{True, False} &
\multicolumn{1}{p{5.0cm}}{Generate fake data (instead of fit)?} \\ \midrule

\multicolumn{1}{p{1.5cm}}{\texttt{generate}} &
\multicolumn{1}{p{2.5cm}}{\texttt{pixelsize}} &
\multicolumn{1}{p{2.0cm}}{2.5} &
\multicolumn{1}{p{2.0cm}}{float $>0.0$} &
\multicolumn{1}{p{5.0cm}}{Pixel size (in units of \texttt{run+bintype}) for the generated wavelength array} \\ \midrule

\multicolumn{1}{p{1.5cm}}{\texttt{generate}} &
\multicolumn{1}{p{2.5cm}}{\texttt{peaksnr}} &
\multicolumn{1}{p{2.0cm}}{0.0} &
\multicolumn{1}{p{2.0cm}}{float $\ge0.0$} &
\multicolumn{1}{p{5.0cm}}{Signal-to-noise ratio (at the peak of the model) for the generated data (0.0 is used for perfect data)} \\ \midrule

\multicolumn{1}{p{1.5cm}}{\texttt{generate}} &
\multicolumn{1}{p{2.5cm}}{\texttt{skyfrac}} &
\multicolumn{1}{p{2.0cm}}{0.0} &
\multicolumn{1}{p{2.0cm}}{float $\ge0.0$} &
\multicolumn{1}{p{5.0cm}}{What is the fractional contribution of the sky (relative to the peak of the model). The condition \texttt{skyfrac} $<$ (peak of model / \texttt{peaksnr}) must hold.} \\ \midrule

\multicolumn{1}{p{1.5cm}}{\texttt{iterate}} &
\multicolumn{1}{p{2.5cm}}{\texttt{model}} &
\multicolumn{1}{p{2.0cm}}{None} &
\multicolumn{1}{p{2.0cm}}{str, None} &
\multicolumn{1}{p{5.0cm}}{Make dynamic changes to the model using a user-specified function ('None' means do not iterate). Two arguments (separated by a comma - no spaces) are allowed, but not necessary. The first argument is the name of the module, the second is any text string that you want passed to the module.} \\ \midrule

%\multicolumn{1}{p{1.5cm}}{\texttt{iterate}} &
%\multicolumn{1}{p{2.5cm}}{\texttt{data}} &
%\multicolumn{1}{p{2.0cm}}{None} &
%\multicolumn{1}{p{2.0cm}}{str, None} &
%\multicolumn{1}{p{5.0cm}}{} \\ \midrule




%\multicolumn{1}{p{1.5cm}}{\texttt{}} &
%\multicolumn{1}{p{2.5cm}}{} &
%\multicolumn{1}{p{2.0cm}}{} &
%\multicolumn{1}{p{2.0cm}}{} &
%\multicolumn{1}{p{5.0cm}}{} \\ \midrule

\end{longtable}
\end{center}

\subsection{The atomic.xml file}

This file may never concern you, but it is good to know what it
does, just in case you ever need it. This file contains all of the
atomic data for a given list of atomic transitions that I think will
be useful to you. It is largely derived from the compilation by
Bob Carswell, and is used in \textsc{vpfit} (with a few updates
that I've found useful). This is a file that is continuously updated
to include the latest laboratory measurements and additional
transitions that others find useful. If you would like to easily view,
update, or add new values to this table, I would recommend
\textsc{topcat}. However, I suggest that you make your own
copy of this file, make changes to it, and load \textit{your}
version of the atomic data file into \textsc{alis}. This can be
done by either changing the default atomic data file listed
in \texttt{settings.alis} (i.e. \texttt{run atomic atomic.xml}), or
you can change this setting in your \dmod\ file. That said, if
you have data for new transitions that I have not included,
(and you think they may also be of use to someone else),
please let me know, and I will update the default version too.
The column data in atomic.xml are as follows:\\

\indent (1) Mass Number\\
\indent (2) Atomic Mass (in amu)\\
\indent (3) Solar Isotopic \textbf{Number/Mass --- TBC} Abundance\\
\indent (4) Element Name\\
\indent (5) Ionization stage\\
\indent (6) Vacuum Wavelength (in \AA)\\
\indent (7) Oscillator Strength\\
\indent (8) Transition Probability (in s$^{-1}$)\\
\indent (9) The $q$-value of the transition (for use with fine-structure constant variation)\\
\indent (10) The $K$-value of the transition (for use with proton-to-electron mass ratio variation)\\

\subsection{The \dmod\ model file}
\label{sec:dmod}

The \dmod\ file contains all of the information that \alis\ needs in order
to perform a fit. In short, you need to include three sections to your
\dmod\ file, the first section is where you can adjust the default settings
of \alis\ (for example, change the number of CPUs that the code uses
to find the best-fitting solution). The second section provides the details
of a given data file that you want \alis\ to read in. The third section contains
the details of the model you wish to fit to the data. There's an optional fourth
section, which tells \alis\ to link a model parameter to another parameter,
through some expression. Therefore, your \dmod\ file should look something
like the following:

\vspace{0.3cm}
\begin{mdframed}[style=MyFrame]
\#\ \alis\ will ignore all text to the right of a `\#' symbol.\\
\#\ This is a comment line.\\

\vspace{0.1cm}

\noindent
\#- -$>$\\
\alis\ will also ignore all script between these
comment \&\ arrow symbols.\\
$<$- -\#\\

\vspace{0.1cm}

\noindent
{\bfseries $<$three parameter arguments$>$}\\

\vspace{0.1cm}

\noindent
data read\\
{\bfseries $<$tell \alis\ where the data are and what to do with it$>$}\\
data end\\

\vspace{0.1cm}

\noindent
model read\\
{\bfseries $<$tell \alis\ what model you want to fit to the data$>$}\\
model end\\

\vspace{0.1cm}

\noindent
link read\\
{\bfseries $<$tell \alis\ what links to enforce between model parameters$>$}\\
link end
\end{mdframed}
\vspace{0.2cm}

\noindent
where the bold arguments in angle brackets are to be chosen from a list
of commands that are described in the following three subsections.

\subsubsection{Three parameter arguments in your \dmod\ file}

The first thing to decide is how you want to change the default settings
that were described in Section \ref{sec:settings}. You would do this in
the same way as you would define/change the commands in the
\texttt{settings.alis} file. For example, if you wanted to change the number of
CPUs that \alis\ uses for the calculation to 8, and you would like
to output the best-fitting model fits and be sure that old fits were
overwritten (without being prompted by \alis), you would need to
include the following three parameter commands at the top of your
\dmod\ file:

\vspace{0.3cm}
\begin{mdframed}[style=MyFrame]
\noindent
run ncpus 8\\
out fits True\\
out overwrite True
\end{mdframed}
\vspace{0.2cm}

\subsubsection{How to read in a data file}
\label{sec:readdata}

Once you've set the details of how \alis\ should run,
you need to specify the data, and a list of the
corresponding properties of that data. The first
argument on every new line must be the pathname
to the datafile (you can either specify the absolute
path [e.g. \texttt{/home/data/datafile.fits}] or the
relative path [e.g. \texttt{../work/datafile.fits}] from the
\dmod\ file). If you specify nothing more on this line,
\alis\ will assign this datafile the default properties
(which is probably not what you want!). It is recommended
that you manually assign the properties of this datafile on
the remainder of the line. The allowed properties you can
set (along with a brief description --- see later for more
details on these properties) are the following:

\begin{itemize}
\item \texttt{specid} --- An ID number that links the data and model (\alis\ will fit data of a given \texttt{specid} with the corresponding model of the same \texttt{specid})
\item \texttt{fitrange} --- The wavelength range of the input data that should be used for the fit
\item \texttt{loadrange} --- The wavelength range of the input data that should be loaded and used to generate the model (\textbf{NOTE:} this command specifies the wavelength range where the model should generated, whereas \texttt{fitrange} tells \alis\ where to calculate the chi-squared)
\item \texttt{resolution} --- Defines the model for the instrumental broadening profile
\item \texttt{shift} --- Defines the model for the relative shift between two different spectra
\item \texttt{systematics} --- define where \alis\ should obtain systematics information from (only if you want to calculate systematics -- see Section~\ref{sec:systematics})
\item \texttt{systmodule} --- A user-defined module to deal with systematics (only if you're doing systematics -- see Section~\ref{sec:systematics})
\item \texttt{columns} --- The columns of the datafile that should be read by \alis\
\item \texttt{loadall} --- Load all of the data (not just the fitted region)
\item \texttt{bintype} --- This parameter allows you to override the default \texttt{run+bintype} for a given set of data
\item \texttt{nsubpix} --- This parameter allows you to override the default \texttt{run+nsubpix} for a given set of data.
\item \texttt{plotone} --- Set to True if you want to plot this data in its own panel (rather than what is specified by \texttt{plot+dims})
\item \texttt{label} --- A label that you would like to have plotted on the lower left corner of the plot
\item \texttt{yrange} --- An optional command if you're unsatisfied with the automatic y-axis range of the plotted data
\end{itemize}

Note that every line
in your \dmod\ file between the \texttt{data read}
and \texttt{data end} commands is considered
independently of all previous lines. This means
that you need to specify the data properties on
every line (or you will just get the default settings).
In general, to set a keyword property type the
keyword followed by an `=' sign, followed by the
value to give that keyword with no spaces! For
example, to set \texttt{specid} to 1, you would
type \texttt{specid=1}. All of the above options
are now described in more detail.

\textbf{\underline{specid}} --- Any model with the corresponding
   \texttt{specid} parameter will be applied to this data. \alis\ reads
   this parameter as a character string, so it can be any value. Note
   that you can specify the same \texttt{specid} for several datafiles.
   In this instance, the model with the corresponding \texttt{specid}
   will be applied to both datafiles. The only instance where this
   parameter is not needed is when you want every specified model
   to be applied to all of the datafiles.

\textbf{\underline{fitrange}} --- This is a required (... well, not really, but it is highly
     recommended that you do!!) parameter
     which tells \alis\ where to fit the model to the data. It can take three arguments.
     The first is the string `all' (which is the default -- without quotation marks), and
     will fit the model to all of the data. This is not recommended if you are convolving
     the data, since your model will contain spurious edge effects. If you want to fit
     only a certain wavelength interval, you can specify the minimum (e.g. 3500.0 \AA)
     and the maximum (e.g. 3600.0 \AA) by typing \texttt{fitrange=[3500.0,3600.0]} where
     the comma is required (no spaces!). The final argument you are allowed to pass in
     is the string `columns' (without quotation marks), which tells \alis\ to obtain the
     wavelengths to be fitted from the datafile itself. If you specify \texttt{fitrange=columns},
     you will need to also specify the column number in the parameter \texttt{columns}
     (see below). The column data needs to be a series of 1's and 0's, where a 1 indicates
     that you would like to include a pixel in the fit, and a 0 tells \alis\ not to use that pixel
     during the fitting process.

\textbf{\underline{loadrange}} --- This is a highly recommended input parameter
     which tells \alis\ over what wavelength range should the model be generated. It can take two arguments.
     The first is the string `all' (which is the default -- without quotation marks, i.e. \texttt{loadrange=all}), and
     will generate a model over the entire wavelength range (this can be slow if you're reading in a large file!).
     This option is not recommended if you are convolving
     the data, since your model will contain spurious edge effects. If you want to generate
     the model within a certain wavelength interval, you can alternatively specify the minimum (e.g. 3480.0 \AA)
     and the maximum (e.g. 3620.0 \AA) by typing \texttt{loadrange=[3480.0,3620.0]} where
     the comma is required (no spaces!).

\textbf{\underline{resolution}} --- This keyword is required, and you should
     always explicitly define it. \alis\ presently has three built-in functions
     for an instrumental profile (\texttt{Afwhm}, \texttt{vfwhm} and \texttt{vsigma}). These functions
     convolve the model with a Gaussian profile: Use \texttt{Afwhm} if you want to
     specifiy the full-width at half-maximum resolution in Angstroms, or use \texttt{vfwhm} if you want to
     specifiy the full-width at half-maximum velocity resolution (i.e. $c/R$), or use
     \texttt{vsigma} is you want to specify the standard deviation. Since the argument
     of this parameter is a function, you will also need to specify the parameters of
     the function. At present, these functions take a single
     argument (which is a constant). For example, if the instrumental resolution is
     known to have FWHM velocity of 7.0 km s$^{-1}$, you should use the command
     \texttt{resolution=vfwhm(7.0)} with no spaces. Since this is a function, the parameters
     of this function are allowed to be a free parameter of the model. In Section~\ref{sec:specifymodel},
     you can find out how to fix, tie, or limit the parameters of this function.
     If you need an instrumental profile that is not built-in, (e.g. if the resolution changes
     linearly with wavelength), or you want to define some arbitrary function for the profile,
     you can do this by (straightforwardly) writing your own function, and loading it into \alis.
     For more details on writing your own functions, see Section~\ref{sec:newfunctions}. The
     \texttt{resolution} parameter can also be set to the string `columns', to read the instrumental
     resolution from a column in the input datafile. In this case, you must specify the velocity
     full-width at half-maximum at every pixel. If you do not want to convolve the data with an instrumental
     profile, just use either \texttt{vfwhm} or \texttt{vsigma} with an argument of 0.0 (and be
     sure to fix the resolution to this value later when specifying the model --- see
     Section~\ref{sec:specifymodel}). In Section~\ref{sec:specifymodel}, there is a
     description for how you can fix and tie the parameters of the instrumental resolution
     profile with uppercase and lowercase letters respectively.

\textbf{\underline{shift}} --- This keyword is not required, but is used to specify (or fit) a relative shift
     between two spectra (for example, if you want to fit for the heliocentric correction).
     \alis\ presently has two built-in functions that allow you to specify the shift
     (\texttt{Ashift}, \texttt{vshift} and \texttt{vsigma}). Use \texttt{Ashift} if you want to
     shift this spectrum by a constant amount in Angstroms, or use \texttt{vshift} if you want to
     set the shift to be a constant velocity shift. Since the argument
     of this parameter is a function, you will also need to specify the parameters of
     the function. At present, these functions take a single
     argument (which is a constant). For example, if the velocity shift is
     estimated to be +30.0 km s$^{-1}$, you should use the command
     \texttt{shift=vshift(30.0)} with no spaces. Since this is a function, the parameters
     of this function are allowed to be a free parameter of the model. In Section~\ref{sec:specifymodel},
     you can find out how to fix, tie, or limit the parameters of this function.
     If you need to specify a shift that is not built-in, (e.g. if the shift changes
     as a function of wavelength), or you want to define some arbitrary function for the shift,
     you can do this by (straightforwardly) writing your own function, and loading it into \alis.
     For more details on writing your own functions, see Section~\ref{sec:newfunctions}.
     In Section~\ref{sec:specifymodel}, there is a
     description for how you can fix and tie the parameters of the shift
     with uppercase and lowercase letters respectively.

\textbf{\underline{systematics}} --- This keyword is to be used if you want to quantify the
     systematics that affect your data. It can (essentially) take one of three arguments for
     this version of \alis. Specifying the string `columns' will tell \alis\ that the relevant
     systematics information is located in one of the columns of the datafile. To set the
     column number of the systematics information, use the \texttt{columns} keyword below.
     You can also specify the string `None' which implies that no `extra' information is required.
     Finally, you can specify the path (absolute or relative), where a datafile with the relevant
     systematics can be found.

\textbf{\underline{systmodule}} --- If none of the built-in systematics routines fit your purpose,
     you can (fairly easily) write your own module to deal with systematics. If you want to do this,
     the argument should provide the name of the \py\ file followed by a comma, followed by a
     string which you can use as an identifier. For example, you would type
     \texttt{systmodule=mysystfile.py,myidstring} to load \texttt{mysystfile.py} with an identifier string
     \texttt{myidstring}. Two arguments, separated by a comma -- with no spaces -- are
     \textit{required} for this parameter. If you want to write your own systematics module,
     please see Section~\ref{sec:montecarlo} for the relevant details.

\textbf{\underline{columns}} --- The columns keyword is required, and informs \alis\ which
     columns of the datafile contain the relevant information. Within square brackets, you can
     include the following information:
     \texttt{wave}, \texttt{flux}, \texttt{error}, \texttt{continuum}, \texttt{fitrange}, \texttt{systematics}.
     \alis\ uses zero-indexed data, so if your wavelength array is the first column of data, you
     will need to specify \texttt{wave:0}. The minimum keywords that you need to specify are
     \texttt{wave}, \texttt{flux}, and \texttt{error} (if you're using \fits\ files, you only need to specify
     the flux and error columns, since the wavelength array is read from the header). If you
     gave the keyword `columns' to either \texttt{fitrange} or \texttt{systematics} (see above),
     then you should specify here which column of data to obtain it from. For example, if the
     wavelength, flux, and error arrays are in the first, second and third columns, whilst the
     fitrange is in the fifth column, you would need to set (without any spaces!) \texttt{columns=[wave:0,flux:1,error:2,fitrange:4]}.
     The keyword \texttt{continuum} will multiply the final model by a fractional amount of the
     continuum. I \textit{do not} recommend that you use your own continuum fits, but rather,
     you should model the continuum self-consistently with \alis\ (yes, you can do this!). The
     \texttt{continuum} option should only be used if you \textit{really} need it.

\textbf{\underline{loadall}} --- \alis\ will try to load only the data that it needs for accurate fitting
     (i.e. it will take the fitted regions you specify plus a bit more so that edge effects from
     convolution does not affect your model profile). If you want to, you can force \alis\ to
     load all of the data in this file (it will slow things down only slightly --- depending on how big
     your data are!). \texttt{True} and \texttt{False} are the only two arguments that are allowed for
     this parameter. To load all of the data, use the command \texttt{loadall=True}.

\textbf{\underline{bintype}} --- This parameter accepts the strings (without quotation marks)
     `km/s' (if your wavelength bins have a constant velocity) or
     `A' (if your wavelength bins have a constant Angstroms).
     This is an optional keyword which, if not specified, will assume the default
     value set by \texttt{run+bintype}. Conversely, if you choose to specify \texttt{bintype}
     for a given datafile, this will override the default setting for \textit{just this datafile}.

\textbf{\underline{nsubpix}} --- This parameter is an optional keyword which is used by
     \alis\ to subpixelate the spectrum (one standard deviation for the model is sampled
     by the number of pixels specified with this keyword). If you do not specify this keyword,
     \alis\ will assume the default value given by \texttt{run+nsubpix}. If you choose to specify
     \texttt{nsubpix} for a given datafile, this will override the default setting for \textit{just this datafile}.

\textbf{\underline{plotone}} --- If you are plotting your results, \alis\ will use the default \texttt{plot+dims}
     arguments toplot your data in different panels. \texttt{plotone} is an optional parameter which allows
     you to override the default plotting panels, and instead plot this datafile and the best-fitting model
     with a single plot. This parameter takes \texttt{True} or \texttt{False} arguments; to plot a given datafile
     in its own window, use the command \texttt{plotone=True}.

\textbf{\underline{label}} --- If you would like to plot a label on a given panel to easily recognize the plot,
     provide a label here, with no spaces and no quotation marks. If you wanted spaces, use the underscore
     character (i.e. $\_$). For example, if you want to highlight that a given panel is Si\,\textsc{ii} $\lambda1808$,
     you would use the command \texttt{label=Si$\_$II$\_$1808}.

\textbf{\underline{yrange}} --- \alis\ will usually derive a reasonable y-axis range for plotting, but you
     can also specify it manually if you prefer. For example, you can specify the minimum value of the
     y plotting axis (e.g. $-0.3$) and the maximum (e.g. $1.5$) by typing \texttt{yrange=[$-0.3,1.5$]} where
     the comma is required (and there are no spaces!).


\subsubsection{How to specify your model}
\label{sec:specifymodel}

Now that the data have been specified, you need to provide the model
that is fitted to the data. The entire model must be specified between the
\texttt{model read} and \texttt{model end} commands. There are several
sections that you need to fill out in order to appropriately specify the model.
The first step is defines the global properties of the model (i.e. what parameters
should be globally fixed (or not), and what the appropriate limits are for the
parameters etc.) In most cases, the default values for the models are sufficient,
but if you want to change them for this \dmod\ file, you can do it here. The next
step is to define the model for emission, then absorption (if you want to fit
absorption features!), and finally, you can perform a fit to the zero-level of the
data (but this feature should probably not be used unless you know
it well (and can fix it to a known value), or you have strong (saturated) absorption
features where you can accurately determine, or fit to, the zero-level). These
sections are now described separately.

\textbf{\underline{A quick note on specifying model parameters}} --- Each model
contains a number of parameters, and a number of keywords. Not all parameters
and keywords are required, but some are. If you don't specify all parameters for
a given model, \alis\ will take the default values (which are sensible -- but may not
be what you're after!). Some keywords are absolutely required (don't worry, \alis\
will flag an error if you don't specify them!). If you want to run a blind analysis
(i.e. you only see the fits and do not see the results) you can globally specify
the \texttt{run+blind} command (this will force the entire run to be
blind --- see Table~\ref{tab:settings}), or you can specify \texttt{blind=True}
on a given model line to blind just that model.

\textbf{\underline{A quick note on fixing and tying model parameters}} --- \alis\
has the ability to tie and/or fix parameters that should be the same. A good example
is a series of Gaussian emission lines that you want to tie together so that they have
the same redshift, and this redshift is a free parameter. In order to tie the redshift of
these two Gaussians together, \alis\ requires a text string that immediately follows
the parameter. If your initial model guess of the redshift is, say, 0.5 then the parameter
you would use could be 0.5tieit (where `tieit' is an arbitrary length (user-defined) string
of lowercase letters only --- you could also use `skfg'). Every model parameter
(with the string `tieit' after the model parameter) that is encountered by \alis\ after
this first instance, will inherit the first instance of `tieit'
(in this case 0.5 --- \underline{regardless} of the model type/parameter).
In other words, `tieit' is thereafter reserved as 0.5, and once fitting has commenced,
all values with `tieit' will change together. If you instead want to fix parameters, use
UPPERCASE letters only. Note that uppercase letters, while fixed, are also tied.
For example, if you had instead set the redshift  above to 0.5R, and you later had
defined a model with the parameter 10.0R, this second model parameter will still
be a fixed number, but it will fixed at 0.5, rather than 10.0. More details (including
examples) for fixing and tying parameters are provided below.

\textbf{\underline{Specifying the emission model}} --- For every set of data that
you want to fit, you must specify a model that describes the emission. For this
example, I will use three Gaussian functions and a constant function. Since it is
emission, it doesn't matter what order we specify these models. Also, we decide
(for a good reason) that we want to tie the redshift for these three gaussians, and
we know that the data are offset by a constant of 1.0. The first thing that needs to
specified is that all of these models are emission. This is achieved by issuing a
line with the text \texttt{emission} (and nothing else). Every subsequent line of
the model will be interpreted as emission. The model definition would be
(explicitly) given by:

\vspace{0.3cm}
\begin{mdframed}[style=MyFrame]
model read\\
emission\\
constant value=1.0CONST\\
gaussian amplitude=9.0 redshift=0.5tval dispersion=1000.0 wave=H\_I\_1215\\
gaussian amplitude=6.0 redshift=0.5tval dispersion=500.0 wave=N\_V\_1238\\
gaussian amplitude=6.0 redshift=0.5tval dispersion=500.0 wave=N\_V\_1242\\
model end
\end{mdframed}
\vspace{0.2cm}

In this example, the value of 1.0 used for the constant is fixed with
the text string `CONST', and the redshift of the three Gaussians are
tied (but free) with the string `tval'. For more details on all of the
built-in models available with \alis, see Section~\ref{sec:functions}. For now,
note that every model has a number of keywords (some are absolutely
required) in addition to the ``required'' parameters (whilst not formally
required, if you don't specify the ``required'' parameters, defaults will
be used, which is not advised!). Default values cannot (or rather, will not)
be used for the keyword arguments. The function called \texttt{constant}
in the above example has a single parameter (called \texttt{value}),
and one keyword argument (called \texttt{specid}).

In the above example, \alis\ thinks you have not specified a
\texttt{specid} for the data, and will therefore apply all four of
these models to all of the data. If you do not wish for this to
happen, simply set the keyword argument \texttt{specid=1}
(with no spaces!) Commas can be used to separate different
values of \texttt{specid}. Therefore, to apply a given model to
the data where \texttt{specid} is 1, 2, and 3, you would use the
keyword argument \texttt{specid=1,2,3} again with no spaces.

To avoid confusion, I have explicitly defined the single parameter
for the \texttt{constant} model as \texttt{value=1.0CONST}. I could
have instead simply used \texttt{1.0CONST} and remove the
`value=' part of the parameter definition. This is because \alis\
expects to read in the model parameters in a certain order. If you
do not explicity tell \alis\ what the parameter name is that a given
number should be assigned to, it will do this automatically. Note
that keywords do not work in the same way. You must specify the
entire keyword as shown above for \texttt{specid}. For further
detail, see Section~\ref{sec:functions} for the order of parameters that is
used in the built-in functions. If you explicitly give the parameter
name, order is not a problem. If you only provide some of the
parameter names explicitly, \alis\ will assign these first, and then
assign the remaining keywords in the appropriate order.

\textbf{\underline{Specifying the absorption model}} --- Once the emission
model is defined, you can then specify the absorption that is superimposed
on top of this emission. The first step is to write the keyword `absorption' on
a new line by itself, followed by the models you wish to define as absorption.
In the following example I will assume there is a damped Lyman-$\alpha$
system providing the absorption on top of the Gaussian emission that I defined
above

\vspace{0.3cm}
\begin{mdframed}[style=MyFrame]
model read\\
emission\\
constant 1.0CONST\\
gaussian amplitude=9.0 redshift=0.5tval dispersion=1000.0 wave=H\_I\_1215\\
gaussian amplitude=6.0 redshift=0.5tval dispersion=500.0 wave=N\_V\_1238\\
gaussian amplitude=6.0 redshift=0.5tval dispersion=500.0 wave=N\_V\_1242\\
absorption\\
voigt ion=1H\_I    20.5   redshift=0.47567RA    4.0DA   1.0E4TA\\
model end
\end{mdframed}
\vspace{0.2cm}

In this particular instance, a DLA with an H\,{\sc i} column density of
$10^{20.5}$ atoms cm$^{-2}$ at redshift $0.47567$, with a turbulent
Doppler parameter of 4.0 km s$^{-1}$ and kinetic temperate of $10^4$ K
will be used. The only free parameter is the column density (which I haven't
explicitly given the parameter name). I have only explicitly given the parameter
name for the redshift (although this wasn't necessary).

In principle, I can now continue to define an emission feature on top of this
absorption feature (for example, Ly$\alpha$ emission in the DLA's trough),
and absorption on top of this emission+absorption+emission (for example,
from another nearby DLA), and so forth. This can be done as follows:

\vspace{0.3cm}
\begin{mdframed}[style=MyFrame]
model read\\
emission\\
constant 1.0CONST\\
gaussian amplitude=9.0 redshift=0.5tval dispersion=1000.0 wave=H\_I\_1215\\
gaussian amplitude=6.0 redshift=0.5tval dispersion=500.0 wave=N\_V\_1238\\
gaussian amplitude=6.0 redshift=0.5tval dispersion=500.0 wave=N\_V\_1242\\
absorption\\
voigt ion=1H\_I    20.5   redshift=0.47567RA    4.0DA   1.0E4TA\\
emission\\
gaussian 0.1 redshift=0.47567RA dispersion=10.0 wave=H\_I\_1215\\
absorption\\
\ldots\\
model end
\end{mdframed}
\vspace{0.2cm}

You might think that this functionality (of being able to indefinitely specify the emission
then absorption etc.) has limited application (and I agree, if you use it for the above
application!). However, a simple (common) example where this functionality might
prove to be useful is if you have several data with different \texttt{specid}. You might
want to use the first set of emission+absorption commands to define the model for
\texttt{specid=1}, the second set of emission+absorption commands to define the
model for \texttt{specid=2}, and so forth. You could equally well define the entire
model for all \texttt{specid} in a single emission+absorption command (although
it would be less well-organised). Usually, I would recommend sticking to a single
set of emission+absorption commands -- it's more guaranteed to work as you expect!

\textbf{\underline{Specifying the zero-level}} --- The zero-level can be fitted for in
the same way that the emission and absorption models are fitted. Simply issue
the command \texttt{zerolevel} on a new line, and follow the same procedures
above. For the current example, we can fit a constant to the zerolevel as follows:

\vspace{0.3cm}
\begin{mdframed}[style=MyFrame]
model read\\
emission\\
constant 1.0CONST\\
gaussian amplitude=9.0 redshift=0.5tval dispersion=1000.0 wave=H\_I\_1215\\
gaussian amplitude=6.0 redshift=0.5tval dispersion=500.0 wave=N\_V\_1238\\
gaussian amplitude=6.0 redshift=0.5tval dispersion=500.0 wave=N\_V\_1242\\
absorption\\
voigt ion=1H\_I    20.5   redshift=0.47567RA    4.0DA   1.0E4TA\\
emission\\
gaussian 0.1 redshift=0.47567RA dispersion=10.0 wave=H\_I\_1215\\
absorption\\
\ldots\\
zerolevel\\
constant 0.01\\
model end
\end{mdframed}
\vspace{0.2cm}

\textbf{\underline{change the default model specifications}} --- You can easily
adjust the limiting values of the models you wish to use for a given fit. To do this,
you can use a series of four parameter arguments. To fix a given parameter from
a model, the first argument is \texttt{fix}. To limit a given parameter from a model,
the first argument is \texttt{lim}. For the second argument, you need to provide the
name of the model function. \alis\ comes built-in with a series of useful functions
(for a full list see Section~\ref{sec:}, and to write your own, see Section~\ref{sec:}).
For the current example we are following, I will use the \texttt{gaussian} function.
This name is used as the second argument. The \texttt{gaussian} function
(as defined by \alis) takes three parameters which are given the (fairly obvious)
name identifiers, \texttt{amplitude}, \texttt{redshift}, and \texttt{dispersion}.
Using the above example, let's assume that we had good reason to expect the
redshift of the emission lines to be exactly (or very close to) 0.5 and we do not
want this to be a free parameter anymore. If you want this parameter to be fixed
(without explicitly changing `tval' to uppercase letters), you can issue the four
argument command \texttt{fix gaussian redshift True} just after the
\texttt{model read} command. Similarly, if you wanted to allow the parameter
\texttt{value} of the function \texttt{constant} to vary (without explicitly changing
`CONST' to lowercase), you would issue \texttt{fix constant value False}.
Other than `True' or `False', you can also use `None' (which will keep the
parameter as is), or you can specify a floating point number that should be
used in place of the current value.

For this example, I'm going to assume that the instrumental resolution profile
was defined by the user as the function \texttt{vfwhm} in the \texttt{data read}
section. This function takes one parameter, called \texttt{value}. To fix (and change)
this value to 4.3, you could issue the command \texttt{fix vfwhm value 4.3}, without
having to explicitly change the \texttt{vfwhm} parameter arguments specified in the
\texttt{data read} section.

As a side note, although the function used as the instrumental profile (see
Section~\ref{sec:readdata} for more details) should not be defined in the
\texttt{model read} section, you can still fix or limit it's value in this section
(in fact, if you want to \textit{change} whether the model parameters from
the instrumental resolution function are fixed or limited, you can only do
it here --- after the \texttt{model read} command). In this case, I will assume

Finally, if you want to limit the amplitude of the Gaussian emission lines
to be between 0.0 and 10.0 you would issue the command
\texttt{lim gaussian amplitude [0.0,10.0]} with no spaces for the final
argument. If you instead feel that the emission should \textit{only} be
limited from below, such that the minimum value is 0.0 and there is
no maximum value, you would instead issue the command
\texttt{lim gaussian amplitude [0.0,None]}, where `None' indicates that
you do not want to specify a limiting value. As such, the model that we
have defined up until this point is the following:

\vspace{0.3cm}
\begin{mdframed}[style=MyFrame]
model read\\
fix gaussian redshift True\\
fix constant value False\\
lim gaussian amplitude [0.0,None]\\
emission\\
constant 1.0CONST\\
gaussian amplitude=9.0 redshift=0.5tval dispersion=1000.0 wave=H\_I\_1215\\
gaussian amplitude=6.0 redshift=0.5tval dispersion=500.0 wave=N\_V\_1238\\
gaussian amplitude=6.0 redshift=0.5tval dispersion=500.0 wave=N\_V\_1242\\
absorption\\
voigt ion=1H\_I    20.5   redshift=0.47567RA    4.0DA   1.0E4TA\\
model end
\end{mdframed}
\vspace{0.2cm}

\textit{Note that all models of a given type \underline{after} the }\texttt{fix}\textit{ and }\texttt{lim}\textit{ declaration
will inherit the }\texttt{fix}\textit{ and }\texttt{lim}\textit{ commands you issue}.
For example, if we now decide that we want to fix the \texttt{dispersion}
parameter for the model function called \texttt{gaussian}, but we only
want to fix the value for the first Gaussian (i.e. with \texttt{dispersion=1000.0}),
without fixing the dispersion for the other two Gaussians
(i.e. with \texttt{dispersion=500.0}), we can place the fix
command at the appropriate location in the \texttt{model read}
section, to free the \texttt{dispersion} parameter for all subsequent
definitions of the \texttt{gaussian} model.

\vspace{0.3cm}
\begin{mdframed}[style=MyFrame]
model read\\
fix gaussian redshift True\\
fix gaussian dispersion True\\
fix constant value False\\
lim gaussian amplitude [0.0,None]\\
emission\\
constant 1.0CONST\\
gaussian amplitude=9.0 redshift=0.5tval dispersion=1000.0 wave=H\_I\_1215\\
fix gaussian dispersion False\\
gaussian amplitude=6.0 redshift=0.5tval dispersion=500.0 wave=N\_V\_1238\\
gaussian amplitude=6.0 redshift=0.5tval dispersion=500.0 wave=N\_V\_1242\\
absorption\\
voigt ion=1H\_I    20.5   redshift=0.47567RA    4.0DA   1.0E4TA\\
model end
\end{mdframed}
\vspace{0.2cm}

If you instead want to place a limit on just a \textit{single} parameter, you can
do so by giving that parameter an ID tag (`\texttt{jc}' in the example below),
and specifying the limit with the command \texttt{lim param jc [20.0,21.0]}
which is of the same form as described above. Similarly, you can fix a single
parameter with the command \texttt{fix param tval True} or free a single
parameter with \texttt{fix param DA False} or fix a parameter to a given
value with \texttt{fix param tval 0.51}.

\vspace{0.3cm}
\begin{mdframed}[style=MyFrame]
model read\\
fix gaussian redshift True\\
fix param tval True\\
fix gaussian dispersion True\\
fix constant value False\\
lim param jc [20.0,21.0]\\
lim gaussian amplitude [0.0,None]\\
emission\\
constant 1.0CONST\\
gaussian amplitude=9.0 redshift=0.5tval dispersion=1000.0 wave=H\_I\_1215\\
fix gaussian dispersion False\\
gaussian amplitude=6.0 redshift=0.5tval dispersion=500.0 wave=N\_V\_1238\\
gaussian amplitude=6.0 redshift=0.5tval dispersion=500.0 wave=N\_V\_1242\\
absorption\\
voigt ion=1H\_I    20.5jc   redshift=0.47567RA    4.0DA   1.0E4TA\\
model end
\end{mdframed}
\vspace{0.2cm}

\textbf{\underline{NOTE:}} Be careful when limiting or fixing
parameters with the same ID tag. If you want to place the same
limit on \textit{two} parameters that are supposed to have different
values, you will need to specify this separately, as follows

\vspace{0.3cm}
\begin{mdframed}[style=MyFrame]
model read\\
lim param jc [20.0,21.0]\\
lim param jd [20.0,21.0]\\
$\vdots$\\
voigt ion=1H\_I    20.7jc   redshift=0.47567RA    4.0DA   1.0E4TA\\
voigt ion=1H\_I    20.2jd   redshift=0.52324RB    3.0DB   1.0E4TA\\
$\vdots$\\
model end
\end{mdframed}
\vspace{0.2cm}

\subsubsection{How to specify your links}
\label{sec:specifylinks}

Links are useful if you know that there is some well-defined relation between two
of your model parameters. For example if you have a good physical reason to believe
that the ratio between two parameters should be fixed, or you want the sum of several
parameters to be a constant value. Linking works very similar to tying parameters, so
you should attach lowercase letters to the right of the parameters in your model that
you want to link.

As a first example, consider the two emission lines [O\,\textsc{iii}]\,$\lambda5007$\,\AA\
and [O\,\textsc{iii}]\,$\lambda4959$\,\AA. In some environments, it is reasonable to assume
that their integrated flux ratio is locked 3:1. To enforce this criteria, we introduce the variable
`\texttt{va}' after specifying the value of the integrated flux for [O\,\textsc{iii}]\,$\lambda5007$\,\AA,
and we also place a \textit{different} variable, in this case `\texttt{vb}', after specifying the value of
the integrated flux for [O\,\textsc{iii}]\,$\lambda4959$\,\AA. Running such a model without
including the \texttt{link} section of the model, would cause both integrated fluxes to be free
parameters. To force their ratio to be 3:1, one must include the link section, explicitly telling
\alis\ that the parameter \texttt{vb} as a function of \texttt{va} is equal to \texttt{va} divide by 3
(i.e. \texttt{vb(va) = va / 3.0}). The model will therefore look something like the following:

\vspace{0.3cm}
\begin{mdframed}[style=MyFrame]
model read\\
emission\\
line\_emission ion=16O\_III\_5007     IntFlux=9.0va         0.03ra       7.0ba\\
line\_emission ion=16O\_III\_4959     IntFlux=3.0vb         0.03ra       7.0ba\\
model end\\

\vspace{0.1cm}

\noindent
link read\\
vb(va) = va / 3.0\\
link end\\
\end{mdframed}
\vspace{0.2cm}

As another example, suppose you wanted to make sure the total integrated flux
from 3 emission lines is equal to 10, you would need to write something like the
model below. In this case, both \texttt{vb} and \texttt{vc} are allowed to vary, but
the variable \texttt{va} will be adjusted such that \texttt{va=10-vb-vc}.

\vspace{0.3cm}
\begin{mdframed}[style=MyFrame]
model read\\
emission\\
line\_emission ion=1H\_I\_6563     IntFlux=5.0va         0.03ra       7.0ba\\
line\_emission ion=1H\_I\_4861     IntFlux=3.0vb         0.03ra       7.0ba\\
line\_emission ion=1H\_I\_4341     IntFlux=2.0vc         0.03ra       7.0ba\\
model end\\

\vspace{0.1cm}

\noindent
link read\\
va(vb,vc) = 10.0 - vb - vc\\
link end\\
\end{mdframed}
\vspace{0.2cm}


\alis\ is fairly able to interpret the expression to the right of the `\texttt{=}' sign, provided that
it contains only numbers (with and without a floating decimal point), and any combination
of the following symbols: \texttt{`+', `-', `*', `/', `**', `(', and `)'}, where the standard order of operations
applies (note that \texttt{`**'} means `to the power of'), and any number of variable strings that are
defined anywhere in the model section. Finally, note that any variable listed on the left hand side
of any linking equation will not be assigned an error. This is because it is related to other parameters,
and by imposing some restriction on this parameter's value, you are removing 1 degree of freedom
from the minimization process. If you specify a link twice for a given parameter (e.g. for the model
above you specified \texttt{va(vb,vc) = 10.0-vb-vc} on one line and \texttt{va(vc) = 5.0-vc} on another
line), the first equation read by \alis\ will be used, and the remaining cases will be ignored.

In the context of linking, the function `\texttt{variable}' can be very useful if you want to calculate
the value and associated error on a combination of model parameters. The `\texttt{variable}'
function takes just one parameter, and this should be tied to a value in the links section of the
model specification. Using the links section can also be useful if you want to start a parameter
value with a random value (drawn from a user-specified distribution).
See Section~\ref{sec:functions} for more details on this functionality.

\textbf{\underline{Note:}} you cannot tie `\texttt{resolution}' variables within \alis.
