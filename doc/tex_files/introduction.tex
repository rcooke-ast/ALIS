\section{Introduction}

If you're looking for software that will help you to fit spectral features,
then you've come to the write place, otherwise, you (probably) won't
want to continue reading. This package was originally written as
``Absorption LIne Software'' (i.e. \alis), but has now been customised
so that you can fit pretty much anything you want (i.e. emission, absorption
or both!). In many aspects (and in other aspects not), \alis\ is similar to code
\textsc{vpfit}, written by Bob Carswell, John Webb, and others. If you're wanting
to perform Voigt profile absorption line fitting using chi-squared minimization,
I would suggest that you download, install and use both \alis\ and \textsc{vpfit}.
They should both (in principle) give you the answer.

This `manual' is designed to be a user (and troubleshooting) guide that will
help you to either install, update, run and develop \alis. There is also a
troubleshooting section (see Section~\ref{sec:troubleshoot}) which you
should consult if you run into any trouble while performing the above actions.

As with all software packages, this one isn't perfect. If you run aground
by trying to do something I (or others) haven't yet tried or tested, then
please contact me. I would happy to try and incorporate your request
in the next version, or suggest how best to get around your problem
in the near future (but please consult the troubleshooting guide first,
as your answer may be there!).

Throughout this guide, you will see several grey text boxes

\vspace{0.5cm}
\begin{myenv}
... a bit like this one
\end{myenv}
\vspace{0.3cm}

These boxes indicate commands that should be issued at
either a terminal, or may indicate code that is used by \alis.
At the time of first writing this documentation, I'm certain that
I've forgotten to list \textit{every} feature that is implemented.
If you feel something could be documented or explained
better, please let me know. Finally, although it is absolutely
not necessary, if you would like to use the code for your work
and want to cite it correctly, please use the following reference:

\vspace{0.2cm}
\noindent
Cooke R. (2013), JOURNAL, REFERENCE.
\vspace{0.2cm}

\noindent
Other than that, thanks for using \alis, and I hope you find it helpful!

\subsection{Software and Documentation License Agreement}

\begin{center}
ALIS --- \textsc{Absorption Line Software}

Copyright \copyright\ \the\year\  Ryan J. Cooke
\end{center}

\noindent
This program is free software: you can redistribute it and/or modify
it under the terms of the GNU General Public License as published by
the Free Software Foundation, either version 3 of the License, or
any later version.

\vspace{0.1cm}

\noindent
This program is distributed in the hope that it will be useful,
but WITHOUT ANY WARRANTY; without even the implied warranty of
MERCHANTABILITY or FITNESS FOR A PARTICULAR PURPOSE.  See the
GNU General Public License for more details.

\vspace{0.2cm}

\noindent
You should have received a copy of the GNU General Public License
along with this program.  If not, see \url{http://www.gnu.org/licenses/}.

\vspace{0.5cm}

\begin{center}
Copyright \copyright\ \the\year\  Ryan J. Cooke
\end{center}

\noindent
Permission is granted to copy, distribute and/or modify the supporting document
under the terms of the GNU Free Documentation License, Version 1.3
or any later version published by the Free Software Foundation;
with no Invariant Sections, no Front-Cover Texts, and no Back-Cover Texts.
A copy of the license is included in the section entitled ``GNU
Free Documentation License''.