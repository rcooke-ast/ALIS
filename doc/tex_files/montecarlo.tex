\section{Monte Carlo Simulations}
\label{sec:montecarlo}

It is also straightforward to perform Monte Carlo simulations with \alis.
You can tell \alis\ to perform Monte Carlo simulations by using the
appropriate three parameter arguments in your \dmod\ file
(see Section~\ref{sec:dmod} and Table~\ref{tab:settings}).
\textbf{NOTE:} This feature is not currently supported, and hasn't
been tested in a *long* time.

\subsection{random}

If you want to perform 1000 Monte Carlo simulations to test the
random error of your data, simply use the three parameter
argument: \texttt{sim random 1000}
at the beginning of your \dmod\ file, \alis\ will do the rest and
save the output appropriately. \textbf{NOTE:} The output will
contain 1001 lines in this case; the first line of the output will
always be the input model parameters (which you can discard
if you don't want them).

In the above example, the Monte Carlo runs will be assigned
the ID number 0--999. If you want to start from a different ID number,
say 10, you would also need to specify \texttt{sim startid 10} (and the
simulations will now run over the ID numbers 10--1010).

For each of the simulations, the best-fitting model parameter file is
output into a directory specified by \texttt{sim+dirname}. If you want
to output the results of your simulations to a directory called
``my\_simulations'', you would issue the following three parameter
argument at the beginning of your \dmod\ file:
\texttt{sim dirname my\_simulations}.

If you decide to find the best-fitting model, and after this you decide that
some Monte Carlo simulations are needed, you do not need to refit the
real data, provided that you have output the best-fitting model file and
covariance matrix. To begin from the best-fitting model, issue the following
three parameter argument \texttt{sim beginfrom myfit.mod.out}. In this
example, from the command line you would need to issue the command
\texttt{alis myfit.mod}. In the same directory that you run this command,
you must have the best-fitting model file (myfit.mod.out), and the corresponding
covariance matrix (myfit.mod.out.covar).

Since fake data is being generated during the Monte Carlo process, your
data may suffer from edge effects if you have convolved your data with the
instrumental resolution. A warning message will be issued if the generated
data suffer from edge effects. You can change the threshold of this warning
using the three parameter argument \texttt{sim edgecut 4.0}, where 4.0 can
be any floating point number. The number you specify represents the number
of standard deviations for the instrumental profile that the fitrange values
need to be from the wavelength edges. If you get these warnings, do not
trust the random simulations. To avoid such warnings, you can either reduce
the \texttt{fitrange}, or provide more wavelength coverage for the input file.
Similarly, you can reduce this threshold (but don't trust your results, unless
you can convince yourself that the edge effects are minimal enough).

\subsection{systematics}

If you have `massaged' your data before running it through \alis\
(for example, you normalise your continuum), and you want to know
how this might systematically affect your results, you can test this with
\alis\ by setting the three parameter argument \texttt{sim systematics True}.
In order to run systematics, you must also run random simulations. The idea
here is that you want to perform the same `massaging' on the data generated
in the random simulations, as you did to the real data. You can either use one
of the built-in systematics modules (in this case, you must match your `massaging'
to the routine that you select), or you can write your own systematics module
(described below). At present, \alis\ only comes built-in with one systematics
modile, which accounts for continuum normalisation with a polynomial.
In order to use this module you will need to issue the three parameter argument
\texttt{sim systmodule default} or you can also issue
\texttt{sim systmodule continuumpoly} for the same routine.
In this case, you must specify an additional column of data to be read in
called \texttt{systematics}. This column of data must contain a $-1$ for every
pixel that is not used in the polynomial fit, and $n$ (where n is unique, and is
the order of the polynomial --- e.g. 1 would be linear) for the pixels that were
used to normalise the continuum. Simlar warnings will be supplied if edge
effects (see above) are important (for example, if you use pixels at the
extremeties of the data to estimate the continuum).

If you would not like to apply systematics to a given datafile that is read in
from the \texttt{data read} section, you can issue \texttt{systmodule=None}
on the appropriate line where you are reading in that datafile.

It is very likely that no other built-in systematics functions will be implemented
(unless you can come up with a fairly common use of this functionality that I can
implement!). I've therefore created the flexibility for you to write your own
systematics module to deal with the systematics that your data may suffer
from. Here are a few details for how to write your own systematics module.
You need to write your module in \python\ in order for it to work.

First, create a \python\ script in the same directory as your \dmod\ file, for example \texttt{mysystmod.py}.
In this file, you can do what you like (!) but you must create a function definition called \texttt{loader}
which accepts several arguments. The first argument is an ID string, which is discussed below. The
second, third, fourth and fifth arguments are respectively the wavelength array, flux spectrum, error spectrum
and the column of data read in from file that has information on the systematics involved (see above).
The sixth argument contains a two element array with the minimum (zeroth element) and maximum
(first element) wavelengths that are free from edge effects (see above for more details). The final
argument is the name of the data file (just incase you need it for reference or whatever).

To tell \alis\ that you want to use your module for a given data file, you need to specify the name of your
module on the corresponding line in the \texttt{date read} section of your \dmod\ file. You can do this by
giving the command \texttt{systmodule=mysystmod.py,idstring}, where `\texttt{,idstring}' is optional, and
can be any text string that you desire. This text string will be passed to the first argument of your
systematics module. If you don't provide an ID string on this line, your systematics module will be passed
a black string with zero characters (i.e. `').


