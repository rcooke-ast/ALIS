\section{Troubleshooting}
\label{sec:troubleshoot}

\subsection{List of Error Messages}

To be completed

\subsection{List of Warning Messages}

To be completed

\subsection{Frequently Asked Questions}

Based on the questions I've received so far (asked by other
users and myself!),I've compiled a list of `frequently asked
questions'. If I've ever encountered a problem, it is usually
listed here to remind me how to solve it! Please look through
here to find if you're problem already has a solution.

\subsubsection{What should the tolerances be?}
Whatever you like! The general rule, is that an \textit{absolute}
change in the $\chi^{2}$ value of 0.001 means that the solution
is converged, and any change in the parameter values are not
physical. Given that \alis\ finds a local minimum, rather than a
global minimum, I would suggest you aim for the above level
in $\chi^{2}$. Therefore, if you want nothing more, set \texttt{xtol}
and \texttt{gtol} to be equal to \texttt{0.0}, and set \texttt{ftol} to be
something small --- of order (or slightly less than) \texttt{0.001/dof},
where \texttt{dof} is the number of degrees of freedom. If you're
running a convergence check, you might want to set \texttt{ftol}
to be \texttt{0.01/dof}, since the convergence check will decrease
\texttt{ftol} by an order of magnitude. Another common practise is
to set \texttt{xtol}, \texttt{gtol}, and \texttt{ftol} to be equal to \texttt{0.0},
and set \texttt{atol} to be 0.001 (which is the tolerance for the
absolute difference between iterations)

\subsubsection{My eye can see a better fit than \alis, what's going wrong?}
There's no simple, gauranteed solution to this problem, but here are a list of
possibilities that you might want to try.

\begin{itemize}

\item Check the reason for convergence (this should be printed in the output file).
If you're doing a blind analysis, the reason for convergence is printed on screen.
If \alis\ has only taken 1 iteration, and then returned, you can be almost certain
that the solution is bad.

\item Check the value of `fstep'. \alis\ works by calculating the
numerical derivatives with the finite difference method. If \texttt{fstep}
is close to 1.0, then the parameters are adjusted by the value of the
parameter multiplied by the machine precision (i.e. hardly at all).
Low values of \texttt{fstep} are preferred, since the derivatives are
better, and the convergence is quicker. If it is too low, some
parameters (depending on how the model is specified), aren't very
sensitive to such a low change (the code thinks that $f$($x$) and
$f$($x+\delta x$) are equal). Try increasing the value of \texttt{fstep}
by an order of magnitude or more.

\item Check the maximum number of iterations is $>1$, and
set the minimum number of iterations to $>1$.

\item Check the maximum number of iterations has not been reached.
If you've reached the maximum number of iterations, the solution has
probably not converged.

\item If \alis\ returns before the minimum number of iterations is reached, then
the machine precision was reached first (unless you received an error message!).
(you probably have a lot of parameters and/or a large dynamic range in these parameters). 
If you have a large number of parameters, try a slightly different set of starting
parameters. Sometimes, if the fit is too good for some parameters and terrible
for others, \alis\ may think it's closer to convergence than it really is.

\item If your model is discretized (i.e. a tabulated model), then the software
can get stuck between two grid points. This can be overcome by interpolation,
or define a model with finer grid sizes.

\item \alis\ does not derive the global minimum, but rather, the local minimum
that is based on the user starting parameters. Try a different set of starting
parameters that is a closer representation to the data.

\end{itemize}

\subsubsection{\alis\ is taking too long to converge}

In general, I really wouldn't expect that your fit will need
more than a couple of hundred iterations (for extreme
examples!). More generally, the number of iterations
should be less than about 100 (and certainly less than
1000!). If \alis\ takes a long time to converge and you're
at less than 100 iterations, you must
either have a complicated model function
(or your model is poorly defined -- make sure you
are using \texttt{numpy} to speed up array operations),
or you have a large number of parameters. Try increasing
the number of cpus that \alis\ uses to speed things up.
However, if you are really doing such a calculation, you
must have a higher-than-average level of patience!

\subsubsection{My model function is not working}

In general, this is something that I'm not able to help you with,
other than suggesting that you read carefully through
Section~\ref{sec:newfunctions}, and make sure that your models
are well-defined (i.e. no discontinuities), and the derivatives can
be numerically calculated fairly easily (for example, absolute
values of free parameters will probably have trouble).

