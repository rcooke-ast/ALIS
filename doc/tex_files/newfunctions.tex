\section{Writing your own function}
\label{sec:newfunctions}

\textbf{XXX --- Ryan, describe here the separate module that should be
used to load user functions}.

\subsection{The Base function}

When writing a function from the bare bones, you need to understand
how \alis\ is written. If you don't care for that (and I don't blame you!),
you can (and should) use the \texttt{Base} functionality to write your
own functions.

\subsection{Writing your own arbitrary function}

Before getting started, if you wish to write a polynomial function,
or a function with an arbitrary number of parameters, you should
read Section~\ref{sec:wrtpoly}.

The best way to learn how to use the \texttt{Base} function is to look at
the \texttt{gaussian} model function that comes built-in with \alis. In
summary, by specifying \texttt{alfunc\_base.Base} as the first and only
argument in the \texttt{python class} environment (see line 6 of the
\texttt{gaussian} model function code), you're function will use all of
the predefined routines that are outlined in \texttt{alfunc\_base.Base},
meaning that all you need to do is write only a few lines of \texttt{python}
code, as discussed below.

The first thing you need to do is change the setup for the model. This
includes changing the text string for what your model function will be
called, how many parameters you wish the model to have, the names and
default values for any keywords that are specified, the names and default
values of the model parameters (including their default limits and if they
should be fixed by default), and the format for how your parameters or
keywords should be printed to screen. You can also specify which keywords
should be printed before the parameters. For an example of the values
you could enter, see lines 14--25 of the \texttt{gaussian} model function.
\textbf{NOTE:} If you are writing your own model function, you
\underline{cannot} use the keyword \texttt{input} as one of your keywords.
This is reserved by \alis\ and cannot be changed.

You will also need to change the models functional form
in the \texttt{def model(par)} section (see lines 41-45 of
the \texttt{gaussian} model function code). Use the variable
\texttt{x} as the wavelength array, and the array \texttt{par}
to define the parameters (\texttt{par[0]} is the first model
parameter, \texttt{par[1]} is the second model parameter
and so forth).

Apart from specifying the functional form of your model, you also
need to tell \alis\ how the input parameters (i.e. the parameters
specified in a user's \dmod\ file) relate to the model parameters
you just specified above in your model function. You should do
this for each of the parameters in your model (see line 63--65
of the \texttt{gaussian} model function for an example).
If you want to combine keywords with model parameters, or you
want to combine several parameters into a single parameter, you
should use the \texttt{parb} variable which is specified in
\texttt{def set\_vars} (see line 78 of the \texttt{gaussian} model
function for an example of passing both a keyword [\texttt{wave}]
and a parameter [the redshift]).

You may also need to change the \texttt{nexbin} parameters (see
the example provided on lines 86--90 of the \texttt{gaussian} model
function) if you want to make sure any subpixellation that's applied
(where \texttt{nexbin[0]}$\equiv$\texttt{run+bintype} and
\texttt{nexbin[1]}$\equiv$\texttt{run+nsubpix}) samples your
model function accurately. For the \texttt{gaussian} model function
that we are using as our example here, two arguments are returned:
The first argument is \texttt{params}, and this should not be changed.
The second argument is the number of subpixels to use for this model.
The definition of \texttt{run+nsubpix} is the number of subpixels per
standard deviation, so \texttt{nexbin[1]/params[2]}, with some corrections
to rounding, gives the desired return value. If you don't wish to implement
this for your function, you can simply return the integer value 1 after
\texttt{params} (i.e. \texttt{return params, 1}).

\subsection{Writing your own polynomial function}
\label{sec:wrtpoly}

This section will help you to write your own arbitrary
polynomial function of your interest. Since polynomials
in \alis\ have the added functionality of specifying
an arbitrary number of coefficients, you can very easily
use the \texttt{Polynomial} base, instead of the \texttt{Base}
base.  All you need to do then is specify the form of the
function that should be multiplied by each of the coefficients
in a definition called \texttt{call\_CPU}.

If you are planning to write your own polynomial function,
I recommend you copy the design of the \texttt{chebyshev}
or \texttt{legendre} model functions that come built-in with \alis,
rather than writing your own polynomial function from scratch.
